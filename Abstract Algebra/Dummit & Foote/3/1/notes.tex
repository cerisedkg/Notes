\documentclass{article}
\usepackage{tikz}
\usepackage{parskip}
\usepackage{xcolor}
\usepackage{textcomp, gensymb}
\usepackage{pgfplots}
\usepackage{tkz-euclide}
\usepackage[bottom=0.5in,top=0.5in,right=0.5in,left=0.5in]{geometry}
\usepackage{amsmath}
\usepackage{amsfonts}
\usepackage{amssymb}
\usepackage{enumitem}
\usepackage{amsthm}
\pgfplotsset{compat=1.18}
\title{3.2: Cosets and Lagrange's Theorem}
\author{Alex L.}
\date{\today}
\pagecolor[rgb]{0,0,0} %black
\color[rgb]{1,1,1} %white

\begin{document}
\maketitle

\textbf{Def:} The \textbf{order} of a finite group is how many elements are in the group. The order is an important group invariant to study.

\textbf{Theorem:} \textbf{Lagrange's Theorem} If $G$ is a finite group and $H \leq G$, then the order of $H$ divides the order of $G$, and the number of cosets of $G/H$ is equal to $\frac{\vert G \vert}{\vert H \vert}$

\textbf{Proof:} Let the order of $H$ be $n$, and the number of left cosets of $H$ in $G$ be $k$. The left cosets of $H$, $gH$, form $k$ disjoint subsets, each with size $n$, so the total size of $G$ is $kn$, therefore, if $\vert H \vert = n$, and $\vert G/H \vert = k$ (because the quotient group is the group of cosets), then $\vert G/H \vert = \frac{\vert G\vert}{\vert H\vert}$.

\textbf{Def:} If $G$ is a group and $H \leq G$, the number of left cosets of $H$ in $G$ is called the \textbf{index} of $H$ in $G$, and is denoted $\vert G:H\vert$.

\textbf{Corrolary:} If $G$ is a finite group and $x \in G$, the order of $x$ divides the order of $G$. Additionally, $x^{\vert G\vert} = 1$ for all $x \in G$

\textbf{Proof:} The order of $x$ is equal to the order of the group generated by $x$, $\vert <x> \vert$. If we let that group equal $H$, then by Lagrange's Theorem, $\vert G \vert$ is a multiple of the order of $x$, meaning the second statement holds.

\textbf{Corrolary:} If $G$ is a group of prime order $p$, then $G$ is cyclic, hence $G \simeq Z_p$

\textbf{Proof:} Cyclic means a group that can be generated by a single element, and by extension, that element has the same order as the entire group. Let $x \in G$ and $x \neq 1_G$. Then, by the previous corrolary, the order of the group generated by $x$ must divide $\vert G\vert$, but it can't be $1$ because $x$ is not the identity. Therefore, since $\vert G \vert$ is prime, $\vert <x> \vert = \vert G \vert$, and the group is cyclic. 

\end{document}