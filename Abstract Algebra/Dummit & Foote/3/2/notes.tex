\documentclass{article}
\usepackage{tikz}
\usepackage{parskip}
\usepackage{xcolor}
\usepackage{textcomp, gensymb}
\usepackage{pgfplots}
\usepackage{tkz-euclide}
\usepackage[bottom=0.5in,top=0.5in,right=0.5in,left=0.5in]{geometry}
\usepackage{amsmath}
\usepackage{amsfonts}
\usepackage{amssymb}
\usepackage{enumitem}
\usepackage{amsthm}
\pgfplotsset{compat=1.18}
\title{3.2: Cosets and Lagrange's Theorem}
\author{Alex L.}
\date{\today}
\pagecolor[rgb]{0,0,0} %black
\color[rgb]{1,1,1} %white

\begin{document}
\maketitle

\textbf{Def:} The \textbf{order} of a finite group is how many elements are in the group. The order is an important group invariant to study.

\textbf{Theorem:} \textbf{Lagrange's Theorem} If $G$ is a finite group and $H \leq G$, then the order of $H$ divides the order of $G$, and the number of cosets of $G/H$ is equal to $\frac{\vert G \vert}{\vert H \vert}$

\textbf{Proof:} Let the order of $H$ be $n$, and the number of left cosets of $H$ in $G$ be $k$. The left cosets of $H$, $gH$, form $k$ disjoint subsets, each with size $n$, so the total size of $G$ is $kn$, therefore, if $\vert H \vert = n$, and $\vert G/H \vert = k$ (because the quotient group is the group of cosets), then $\vert G/H \vert = \frac{\vert G\vert}{\vert H\vert}$.

\textbf{Def:} If $G$ is a group and $H \leq G$, the number of left cosets of $H$ in $G$ is called the \textbf{index} of $H$ in $G$, and is denoted $\vert G:H\vert$.

\textbf{Corrolary:} If $G$ is a finite group and $x \in G$, the order of $x$ divides the order of $G$. Additionally, $x^{\vert G\vert} = 1$ for all $x \in G$

\textbf{Proof:} The order of $x$ is equal to the order of the group generated by $x$, $\vert <x> \vert$. If we let that group equal $H$, then by Lagrange's Theorem, $\vert G \vert$ is a multiple of the order of $x$, meaning the second statement holds.

\textbf{Corrolary:} If $G$ is a group of prime order $p$, then $G$ is cyclic, hence $G \simeq Z_p$

\textbf{Proof:} Cyclic means a group that can be generated by a single element, and by extension, that element has the same order as the entire group. Let $x \in G$ and $x \neq 1_G$. Then, by the previous corrolary, the order of the group generated by $x$ must divide $\vert G\vert$, but it can't be $1$ because $x$ is not the identity. Therefore, since $\vert G \vert$ is prime, $\vert <x> \vert = \vert G \vert$, and the group is cyclic. 

\textbf{Ex:} Let $H$ be a subgroup of $G$ with $H$ in $G$ having an index of $2$ (there are two cosets of $H$ in $G$). Then, we will prove that $H$ is normal in $G$

\textbf{Proof:} Let $g \in G-H$, then, we have two subgroups, $1H$ and $gH$, which together, partition $G$. $1H = H1$, so therefore, $gH = Hg$, and as we proved in section 3.1, $gH = Hg$ indicates $H$ is normal in $G$. 

\textbf{Theorem:} If $G$ is a finite group and $p$ is a prime which divides $\vert G \vert$, then there is an element of order $p$ in $G$

\textbf{Theorem:} If $G$ is a finite group of order $p^\alpha m$, where $p$ is a prime and $p$ doesn't divide $m$, then $G$ has a subgroup of order $p^\alpha$. 

\textbf{Def:} Let $H,K$ be subgroups and define $$HK = \{hk \ \vert\ h \in H, k\in K \}$$

\textbf{Prop:} If $H$ and $K$ are finite subgroups, then $\vert HK \vert = \frac{\vert H \vert \vert K \vert}{\vert H \cap K\vert}$.

\textbf{Proof:} $HK$ is actially the set of left cosets of $K$ with elements of $H$. We want to find how many distinct left cosets of $K$ there are. 

\textbf{Prop:} If $H$ and $K$ are subgroups of a group, then $HK$ is a subgroup if and only if $HK - KH$.

\textbf{Proof:} $HK$ is the group of one element of $H$ multiplied by another element from $K$. Forward Proof: Let $h_1k_1, h_2k_2$ be elements in $HK$. $HK$ has an identity because both $H$ and $K$ have an identity. We want to show that $h_1k_1(h_2k_2) ^{-1} $ is in $HK$. First, note that $(h_2k_2) ^{-1} $ is equal to $k_2 ^{-1} h_2 ^{-1}$. Substituting, we get $h_1k_1k_2 ^{-1} h_2 ^{-1}$. Since $K$ is a group, $k_1 k_2 ^{-1} $ is equal to another element in $K$, $k_3$. Substituting, we get $h_1 k_3 h_2 ^{-1}$. Since $HK = KH$, for every element $kh$ in $KH$, there is a corresponding $hk$ in $HK$. As such, $k_3 h_2 ^{-1} $ is equal to $h_3 k_4$ in $HK$. Then, we get $h_1 h_3 k_4$, and since $H$ is a group, the equation evaluates to $h_4 k_4$. This is obviously in $HK$, so our proof is done. 

Reverse Proof: We want to show that given the conditions that $HK$ is a group, then $HK = KH$. To show equality of groups, we need to show they are subgroups of each other. Since $K \leq HK$ because $1_H k$, and $H \leq HK$, we know that $KH \subseteq HK$. To show the opposite, note that $hk = (h_1 k_1) ^{-1}  =  k_1 ^{-1} h_1 ^{-1}$, which is in $KH$, so every element of $HK$ is in $KH$. We have established our two inclusions, so these sets are equal. 

\textbf{Corrolary:} If $H$ and $K$ are subgroups of $G$ and $H \leq N_G(K)$, then $HK$ is a subgroup of $G$. If $K \trianglelefteq G$ then $HK \leq G$ for any $H \leq G$.

\textbf{Proof:} Since $H$ normalizes $K$, $hk$ is in $KH$ because $kh = hkh ^{-1} h = hk$. Also, $kh$ is in $HK$ because $kh = h h ^{-1} kh = hk$. 

\subsection{Exercises:}

\begin{enumerate}
    \item Which of the following are permissible orders for subgroups of order $120$: $1,2,5,7,9,15,60,240$, and what is the index of each order? \\ \textbf{Solution:} The order of a subgroup must divide the order of the parent group. The index is the order of the parent group divided by the order of a subgroup. $(\text{order},\text{index}): (1,120), (2,60), (5,12), (15,8), (60,2)$.
    \item Prove the lattice of subgroup of $S_3$ is correctly drawn in Section 2.5. \\ \textbf{Solution:} We want to show that all of the nodes are actually subgroups of $S_3$ and their parents. The order of $S_3$ is 6, and we have four subgroups drawn: $(1,2,3)$, $(1,2),(2,3),(1,3)$. The order of these is $3,2,2,2$ respectively. They are all indeed subgroups. None of them link to each other, which is also correct, so they are drawn correctly. 
    \item Proce the lattice of subgroups of $Q_8$ is correctly drawn in Section 2.5. \\ \textbf{Solution:} The quaternion group has an order of $8$, and the drawn subgroups are $(i),(j),(k),(-1)$, having orders of $4,4,4,2$ respectively, and $(-1)$ is a group of the former three, which all checks out.
    \item Show that if $\vert G \vert = pq$ for some primes $p$ and $q$, then either $G$ is abelian or $Z(G) = 1$ \\ \textbf{Solution:} The center of a group, $Z(G)$ is always a subgroup of $G$, so by Lagrange's Theorem, $Z(G)$ has order $1,p,q$ or $pq$. If $Z(G)$ has order $p$ or $q$, then it has index $q$ or $p$ respectively. Therefore, $G/Z(G)$ has a prime order, and by the corrolary above, it will be cyclic and therefore abelian. If the quotient group is abelian, so is the main group, so $G$ is abelian. The only other cases are when $Z(G)$ has order $pq$, which means $Z(G) = G$ and so it is abelian, or when $Z(G)$ has order $1$, in which case it is the trivial group, $\{1\}$, because it is the only group with order $1$. 
\end{enumerate}
\end{document}