\documentclass{article}
\usepackage{tikz}
\usepackage{parskip}
\usepackage{xcolor}
\usepackage{textcomp, gensymb}
\usepackage{pgfplots}
\usepackage{tkz-euclide}
\usepackage[bottom=0.5in,top=0.5in,right=0.5in,left=0.5in]{geometry}
\usepackage{amsmath}
\usepackage{amsfonts}
\usepackage{amssymb}
\usepackage{enumitem}
\usepackage{amsthm}
\pgfplotsset{compat=1.18}
\title{3.3: The Isomorphism Theorems}
\author{Alex L.}
\date{\today}
\pagecolor[rgb]{0,0,0} %black
\color[rgb]{1,1,1} %white

\begin{document}
\maketitle

\textbf{Theorem:} \textbf{The First Isomorphism Theorem:} If $\varphi : G \rightarrow H$ is a homomrophism of groups, then $\ker \varphi \trianglelefteq G$ and $G/ \ker \varphi  \simeq \varphi G$

\textbf{Corrolary 17:} Let $\varphi : G \rightarrow H$ be a homomorphism of groups.
\begin{enumerate}
    \item $\varphi $ is injective iff $\ker \varphi = 1$
    \item $\vert G : \ker \varphi \vert = \vert \varphi (G) \vert$
\end{enumerate}  

\textbf{Theorem:} \textbf{The Second Isomorphism Theorem} Let $G$ be a group, let $A$ and $B$ be subgroups of $G$ and assume $A \leq N_G(B)$. Then, $AB$ is a subgroup of $G$, $B \trianglelefteq AB$, $A \cap B \trianglelefteq A$, and $AB/B \simeq A/(A\cap B)$. (Remember that $N_G(A)$ is the set of elements in $G$ that commute with all elements in $A$)

\textbf{Proof:} Note: all elements of $A$ do normalize $B$. Then, by definition, $aba ^{-1}$ is in $B$. Therefore, $a (aba ^{-1})$ is in $AB$. 

\end{document}