\documentclass{article}
\usepackage{tikz}
\usepackage{parskip}
\usepackage{xcolor}
\usepackage{textcomp, gensymb}
\usepackage{pgfplots}
\usepackage{tkz-euclide}
\usepackage[bottom=0.5in,top=0.5in,right=0.5in,left=0.5in]{geometry}
\usepackage{amsmath}
\usepackage{amsfonts}
\usepackage{amssymb}
\usepackage{enumitem}
\usepackage{amsthm}
\pgfplotsset{compat=1.18}
\title{3.4: Composition Series and the H\"older Program}
\author{Alex L.}
\date{\today}
\pagecolor[rgb]{0,0,0} %black
\color[rgb]{1,1,1} %white

\begin{document}
\maketitle

\textbf{Prop:} If $G$ is a finite abelian group, and $p$ is a prime which divides $\vert G \vert$, then $G$ contains an element of order $p$.

\textbf{Proof:} We will split this into some cases: Case 1: $\vert G \vert = p$. $p$ must be greater than $1$ if there is a prime number that divides it, and by the corrolary to Lagrange's Theorem, if $\vert G \vert$ is prime, $G$ is cyclic, so there is an element with order $p$. Case 2: $\vert G \vert > p$. Lets suppose $p$ then divides the order of some nonidentity $x$ in $G$, so $\vert x \vert = pn$. Then, $\vert x^n \vert = p$, so $p$ can't divide the order of any element. Now, let $N$ be the group generated by an element $x$. Since $G$ is abelian, any subgroup of $G$ is normal, so $N \trianglelefteq G$. Then, by Lagrange's theorem, $\vert G/N\vert = \frac{\vert G \vert}{\vert N \vert}$, and since $\vert N \vert \neq 1$ (because $x \neq 1$), $\vert G/N \vert < \vert G \vert$. $p$ does not divide $\vert N\vert$, but $p$ divides $\vert G\vert$, so $p$ must divide $\vert G/N\vert$. Now, we can apply an induction assumption that the smaller group $G/N$ does have an element of order $p$ (we proved a base case of size $p$ above, so we can do this). This element, $yN$, has the property that $\vert y^p\vert = 1$, but since $y \neq N$, $\vert y\vert > 1$, so $\vert y^p \vert < \vert y\vert$, but this means that $p$ divides $\vert y\vert$, so we are back at our scenario with $x$, and our proof is solved. 

\textbf{Def:} A group $G$ is called simple if $\vert G\vert > 1$ and the only normal subgroups of $G$ are $1$ and $G$.

If $\vert G\vert$ is prime, then its only subgroups are $1$ and $G$, so it is simple. Every abelian simple group is isomorphic to $\mathbb{Z}_p$, but there are nonabelian simple groups, the smallest of which has order $60$. 

\textbf{Def:} In a group $G$, a sequence of subgroups $1 \leq N_1 \leq N_2 \leq N_3 \leq ... \leq N_{k-1} \leq G$ is called a composition series if each group is normal in the succeeding group, and $N_{i+1}/N_i$ is simple, for $0 \leq i \leq k-1$, is called a \textbf{composition series}, and the quotient groups $N_{k+1}/N_k$ are called \textbf{composition factors}.

\textbf{Ex:} $1 \trianglelefteq <s> \trianglelefteq <s,r^2> \trianglelefteq D_8$. The dihedral $8$ group corresponds to rotations and flips of a square. $s$ is a flip, and $r$ is a rotation. As we can see, $<s> = \{1,s\}$, and is clearly normalized by $1$. In addition, $<s>/1 = \{1,s\}$, which has no other normal subgroups. $<s,r^2> = \{1,s,r^2, r^2s\}$. It is normalized by $<s>$ because $r^2s = sr^2$, $r^2sr^2 = r^2r^2s$, and $r^2r^2 = r^2r^2$ and $1$ obviously normalizes everything else. It is simple because $<r^2,s>/<s> = \{\{1,s\},\{r^2, s^2\}\}$, and the only possible subgroups are $1$ and $G$. Finally, $<s,r^2>$ is normal in $D_8$ because $D_8$ is abelian, and $D_8 / <s,r^2> = \{<s,r^2>,r<s,r^2>\}$, and only has two elements, so it is simple. 

\textbf{Theorem:} \textbf{Jordan H\"older Theorem:} Let $G$ be a finite group with $G \neq 1$. Then:
\begin{enumerate}
    \item $G$ has a composition series
    \item The composition factors in a composition series are unique, namely, if $1 \leq N_1 \leq ... \leq N_r = G$ and $1 \leq M_1 \leq ... \leq N_r = G$, then, there is some arrangement of the first series you can make such that $M_i / M_{i-1} \cong N_i/ N_{i-1}$ for all $1\leq i \leq r$. (Basically, although the series are different, composition factors are unique)
\end{enumerate}

\textbf{The H\"older Program:} The H\"older Program is a method for finding and classifying every single group that can exist. It has two steps:
\begin{enumerate}
    \item Classify all finite simple groups
    \item Combine simple groups together to form other groups
\end{enumerate}  

The first part of The H\"older Program has been completed, and all simple groups have been found. 

\textbf{Theorem:} There is a list containing 18 infinite families of simple groups and 26 simple groups not belonging to these families, and every other finite simple group is isomorphic to one on the list

\textbf{Theorem:} If $G$ is a simple group with odd order, then $G \cong Z_p$ for some prime $p$. 

\textbf{Def:} A group $G$ is solvable if there is a chain of subgroups $1=G_0 \trianglelefteq G_1 \trianglelefteq G_2 \trianglelefteq ... \trianglelefteq G_s = G$ such that $G_{i+1}/G_i$ is abelian for $i = 0,1,...,s-1$

\textbf{Theorem:} The finite group $G$ is solvable if and only if for every divisor $n$ of $\vert G\vert$ such that $n$ and $\frac{\vert G\vert}{n}$ are coprime, $G$ has a subgroup of order $n$. 

\textbf{Theorem:} If $N$ and $G/N$ are solvable, then $G$ is solvable. 

\textbf{Proof:} Let $\overline {G} = G/N$, let $1 = N_0 \trianglelefteq N_1 \trianglelefteq N_2 \trianglelefteq ... \trianglelefteq N_n = N$ be a chain of subgroups of $N$ such that $N_{i+1}/N_i$ is abelian. Let $\overline{1} = \overline{G_0} \trianglelefteq \overline{G_1} \trianglelefteq \overline{G_2} \trianglelefteq ... \trianglelefteq \overline{G_m} = \overline{G}$ be a chain with the same properties. There are subgroups $G_i$ of $G$ with $N \leq G_i$ such that $G_i/N = \overline{G_i}$ and $G_i \trianglelefteq G_{i+1}$. Then, by the third isomorphism theorem, $\overline{G_{i+1}}/\overline{G_i} \cong G_{i+1}/G_i$, thus, $1 = N_0 \trianglelefteq N_1 \trianglelefteq ... \trianglelefteq N_n = N = G_0 \trianglelefteq ... \trianglelefteq G_m = G$ is a valid chain of subgroups with successive abelian compoisiton factors.      
\end{document}