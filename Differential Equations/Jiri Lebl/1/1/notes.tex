\documentclass{article}
\usepackage{tikz}
\usepackage{parskip}
\usepackage{xcolor}
\usepackage{textcomp, gensymb}
\usepackage{pgfplots}
\usepackage{tkz-euclide}
\usepackage[bottom=0.5in,top=0.5in,right=0.5in,left=0.5in]{geometry}
\usepackage{amsmath}
\usepackage{amsfonts}
\usepackage{amssymb}
\usepackage{enumitem}
\usepackage{amsthm}
\pgfplotsset{compat=1.18}
\title{1.1: Integrals as Solutions}
\author{Alex L.}
\date{\today}
\pagecolor[rgb]{0,0,0} %black
\color[rgb]{1,1,1} %white

\begin{document}
\maketitle

\textbf{Def:} The general form of a first order ordinary differential equation is $\frac{dy}{dx} = f(x,y)$

\textbf{Technique:} When a first order ODE takes the form $\frac{dy}{dx} = f(x)$, we can integrate both sides with respect to $x$ to get $y(x) = \int{f(x) dx} + C$. $y(x)$ is the general solution for the ODE.

\textbf{Technique:} If we are given an \textbf{initial value problem} (IVP) with starting values $y(x_0) = y_0$, we can directly solve for the particular solution by adding a lower bound of integration $x_0$ and an upper bound of integration $x$, and adding $y_0$ outside the integral. The particular solution ca be found by the formula $y(x) = \int_{x_0}^{x} f(x) dx + y_0 = (f(x) - f(x_0)) + y_0$

\textbf{Technique:} If we are given a first order ODE in the fomr $\frac{dy}{dx} = f(y)$, we can swap the roles of the dependent and independent variable by taking the reciprocal of both sides. We get $\frac{dx}{dy} = \frac{1}{f(y)}$. By integrating, we get $x(y) = \int{\frac{1}{f(y)} dy} + C$. Then, simply rewrite the resulting equation in terms of $x$. Keep in mind that this change of variables only works if the function $f(y)$ is invertible, meaning there exists a well defined inverse $f^{-1}(y)$.

\textbf{Exercises:}  

$1.1.2.$    Solve $\frac{dy}{dx} = x^2 + x$ for $y(1) = 3$ \\ \textbf{Solution:} $y(x) = \int_1^x x^2 + x \ dx + 3 = (\frac{1}{3}x^3 + \frac{1}{2}x^2 - \frac{5}{6}) + 3 = \frac{1}{3} x^3 + \frac{1}{2} x^2 + \frac{13}{6}$



\end{document}