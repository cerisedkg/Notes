\documentclass{article}
\usepackage{tikz}
\usepackage{parskip}
\usepackage{xcolor}
\usepackage{textcomp, gensymb}
\usepackage{pgfplots}
\usepackage{tkz-euclide}
\usepackage[bottom=0.5in,top=0.5in,right=0.5in,left=0.5in]{geometry}
\usepackage{amsmath}
\usepackage{amsfonts}
\usepackage{amssymb}
\usepackage{enumitem}
\usepackage{amsthm}
\pgfplotsset{compat=1.18}
\title{1.2: Slope Fields}
\author{Alex L.}
\date{\today}
\pagecolor[rgb]{0,0,0} %black
\color[rgb]{1,1,1} %white

\begin{document}
\maketitle

\subsection{Slope Fields}

A general first order ODE looks like $\frac{dy}{dx} = f(x,y)$. By plugging in values for $x,y$, $f(x,y)$ tells you the slope of $y(x)$ at every point in the plane. If we draw a line with the slope that $f(x,y)$ gives us at every point in the plane, we have made a slope field.

To find a particular solution to a slope field, just pick a starting point, and draw a curve that is tangent to every line you encounter on the slope field.

\subsection{Existence and Uniqueness}

\textbf{Def:} A solution \textbf{exists} if there is a valid solution at a given point.

\textbf{Def:} A solution is \textbf{unique} if there is only one solution at that point.

\textbf{Theorem:} Picard's Theorem on Existence and Uniqueness: If $\frac{dy}{dx} = f(x,y)$ is defined at $(x_0,y_0)$ and $\frac{\partial f}{\partial y}$ is also defined at $(x_0,y_0)$, then there exists a unique solution at and around $(x_0,y_0)$.
\end{document}