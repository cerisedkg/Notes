\documentclass{article}
\usepackage{alexconfig}
\title{2.4: Mechanical Vibrations}

\begin{document}
\maketitle

\begin{proposition}[Damped Harmonic Oscillators]
Consider a mass $m$ on a spring with spring constant $k$, and let the distance from equilibrium be $x$. Also, lets suppose there is some other force opposing the motion of the mass, $c \frac{dx}{dt}$, which we call the \textbf{damping force}. There also might be some external force $F_{ex}(t)$ on the spring as well. Lets sum all of the forces and get an equation. The sum of the forces, $F_{ex}(t) - c \frac{dx}{dt} - kx = m \frac{d^2x}{dt^2}$, is equal to mass times acceleration. Rearranging, we get $$m \frac{d^2x}{dt^2} - c \frac{dx}{dt} - kx = F(T)$$
\end{proposition}
\begin{definition}
Some common terminology:
    \begin{enumerate}
        \item When $F(t) = 0$ for all $t$, the system is \textbf{free}
        \item When $F(t) \neq 0$ for all $t$, the system is \textbf{forced}
        \item When $c > 0$ , the system is \textbf{damped}
        \item When $c = 0$, the system is \textbf{undamped}    
    \end{enumerate} 
\end{definition}

\subsection{Free Undamped Motion}

\begin{definition}[Undamped Harmonic Oscillation]
The equation describing undamped harmonic oscillators is:$$m \frac{d^2x}{dt^2}  + kx = 0$$
\end{definition}
Lets divide through by $m$ to get $$\frac{d^2x}{dt^2} - \frac{k}{m} x = 0$$.

By setting $\omega_0 = \sqrt{\frac{k}{m}}$ we get $$\frac{d^2x}{dt^2} - \omega_0^2x = 0$$

The general solution to the above equation is $$x(t) = A\cos(\omega_0 t) + B\sin(\omega_0 t)$$which is equal to $$x(t) = C\cos(\omega_0 t - \gamma)$$for some constant $C$ and $\gamma$.

Through some algebra, we arrive at $C = \sqrt{A^2 + B^2}$ and $\tan\gamma = \frac{B}{A}$

\subsection{Free Damped Motion}

\begin{definition}[Free Damped Harmonic Oscillation]
The equation describing free damped harmonic oscillation is:
$$m\frac{d^2x}{dt^2} + c\frac{dx}{dt} + kx = 0$$
\end{definition}

Now, lets divide through by $m$ to get $\frac{d^2x}{dt^2} + \frac{c}{m} \frac{dx}{dt} + \frac{k}{m} x = 0$Let set variables $\gamma = \frac{c}{2m}$ and $\omega_0 = \sqrt{\frac{k}{m}}$

Substituting, we get $$\frac{d^2x}{dt^2} + 2\gamma \frac{dx}{dt} + \omega_0^2 = 0x = 0$$and since the mass, spring constant, and damping factor probably aren't changing, this is a linear homogeneous constant coefficients second order ODE.

The characteristic equation is $$r^2 + 2\gamma r + \omega_0^2 = 0$$treating $\omega_0^2$ as a single variable.

The roots of this equation are $r = -\gamma\pm\sqrt{\gamma^2-\omega_0^2}$. The sign of the determinant, $\gamma^2 - \omega_0^2$, is the same as $c^2 - 4km$, so we get real roots only if $c^2 \geq 4km$.

\begin{definition}[Overdamping]
If we have two real roots, the system is \textbf{overdamped}, and the solution becomes $$x(t) = C_1e^{r_1t} + C_2e^{r_2t}$$$r_1$ and $r_2$ are both negative because $\gamma$ is always greater than $\sqrt{\gamma^2 - \omega_0^2}$. Over time, the motion in the system will approach zero. 
\end{definition}

\begin{definition}[Critical Damping]
If $c^2 = 4km$, the system is \textbf{critically damped}, and there is one root with multiplicity $2$. Solutions look like:$$x(t) = C_1e^{r_1t} + C_2te^{r_1t}$$
\end{definition}

\begin{definition}[Underdamped]
If $c^2 < 4km$, the system is \textbf{underdamped}  , and we have two complex roots. The solution becomes $$x(t) = Ce^{-\gamma t}(A\cos(\omega_1 t) + B\sin(\omega_1 t))$$Our system develops limitng constraints specifying the maximum value $x(t)$ can take at any given time, and stays the same even under phase shifts. This constraint is called the \textbf{envelope curve}. 
\end{definition}

\end{document}