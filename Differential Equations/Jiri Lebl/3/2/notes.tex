\documentclass{article}
\usepackage{alexconfig}
\title{3.2: Matrices and Linear Systems}

\begin{document}
\maketitle
\begin{definition}[Matrices]
A \textbf{matrix} $A$ is an $m$ by $n$ array of numbers ($m$ rows and $n$ columns). The numbers $A_{ij}$ refer to the entry in the $i$th row and $j$th column of a matrix. 
\end{definition}

\begin{definition}[Multiplication of Matrices]
The component $(AB)_{ij}$ is found by taking the dot product of the $i$th row of matrix $A$ with the $j$th column of matrix $B$. 
\end{definition}

\begin{definition}[Determinant]
The \textbf{determinant} of a 2x2 square matrix $\text{det} A$ is calculated by $a_{11}a_{22} - a_{21}a_{12}$. The determinant of an $n$ by $n$ square matrix represents that if a space were to be transformed by that matrix (basis vectors get mutliplied by matrix entries), then the area/volume of shapes would be scaled by the determinant. Negative determinant values means that the shapes get flipped/mirrored. 
\end{definition}

$$\text{det}(AB) = \text{det}(A) \text{det}(B)$$

\subsection{Solving Linear Systems}

Suppose we have a system of linear equations $$2x_1 + 2x_2 + 2x_3 - 2$$$$x_1 + x_2 + 3x_3 = 5$$$$x_1+4x_2+x_3 = 10$$We can represent this system as an augmented matrix: $$\left[\begin{array}{@{}ccc|c@{}}
    2 & 2 & 2 & 3 \\
    1 & 1 & 3 & 5 \\
    1 & 4 & 1 & 10 \\
    \end{array}\right]$$

We can use some rules to put this in reduced row echelon form:
\begin{enumerate}
    \item 1. You can multiply any row by a scalar
    \item 2. You can add a multiple of one row to another row
    \item 3. You can swap two rows
\end{enumerate}

\begin{definition}[Row Reduced Echelon Form]
A matrix is in \textbf{row reduced echelon form} if it satisfies some criteria:

\begin{enumerate}
    \item The \textbf{leading entry}, or the first entry in any row, is strictly to the right of the leading entry of the row above
    \item Any zero rows are below all nonzero rows
    \item All leading entries are 1
    \item All entries above and below any leading entry are zero.  
\end{enumerate}

If every column has a leading entry in RREF, then the matrix is \textbf{invertible}, or an inverse matrix exists. If there are any free variables, or columns without leading entries, then the matrix is not invertible.  
\end{definition}

\subsection{Computing the Inverse}

If we want to find the inverse of a matrix $A$, simply construct an augmented matrix $[A\vert I]$, and then put the left hand side of the matrix into RREF, so we get $[I\vert A^{-1}]$, and the right hand side will be the inverse.
\end{document}