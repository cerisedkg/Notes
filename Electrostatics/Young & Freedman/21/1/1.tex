\documentclass{article}
\usepackage{tikz}
\usepackage{parskip}
\usepackage{xcolor}
\usepackage{tkz-euclide}
\usepackage[bottom=0.5in,top=0.5in,right=0.5in,left=0.5in]{geometry}
\usepackage{amsmath}
\usepackage{amsfonts}
\usepackage{amssymb}
\usepackage{amsthm}
\title{21.1: Electric Charge}
\author{Alex L.}
\date{\today}
\pagecolor[rgb]{0,0,0} %black
\color[rgb]{1,1,1} %white

\begin{document}
\maketitle

Electromagnetic interactions involve particles with electric charges, which is a fundamental attribute of matter. 

\textbf{Def:}  Two same charged particles will repel each other, while two different charged particles will attract each other. This attractive / repulsive force is called the \textbf{electromagnetic force}.

The most fundamental charged particles are quarks, the building blocks of matter. Quarks have either $\pm \frac{1}{3}$ or $\pm \frac{2}{3}$ charge. These quarks make up charged particles called electrons and protons. An electron has a mass of $9.109 * 10^{-31} kg$ and a proton has a mass of $1.673 * 10^{-27} kg$. Electrons are negatively charged, and protons are positively charged.

\textbf{The Conservation of Charge}: The sum of all electric charges in any closed system is constant.

Charging an object really means transferring electrons from one object to another. The object losing electrons develops a positive charge, and the object gaining electrons develops a negative charge.

\textbf{Quantization of Charge} The smallest unit of charge that can exist in an independent particle is the magnitude of charge of a proton/electron.

Quarks can have fractional charges, but have never been observed independently, only bonded in groups that make a whole charge.

\end{document}