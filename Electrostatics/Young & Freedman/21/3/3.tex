\documentclass{article}
\usepackage{tikz}
\usepackage{parskip}
\usepackage{xcolor}
\usepackage{tkz-euclide}
\usepackage[bottom=0.5in,top=0.5in,right=0.5in,left=0.5in]{geometry}
\usepackage{amsmath}
\usepackage{amsfonts}
\usepackage{amssymb}
\usepackage{amsthm}
\title{21.3: Coulomb's Law}
\author{Alex L.}
\date{\today}
\pagecolor[rgb]{0,0,0} %black
\color[rgb]{1,1,1} %white

\begin{document}
\maketitle

\textbf{Coulomb's Law} $F_e = k \frac{\vert q_1 \cdot q_2 \vert}{r^2}$

Coulomb's Constant: $k = \frac{1}{4 \pi \varepsilon_0} = 8.99 \cdot 10^9 \frac{N \cdot m^2}{C^2}$, where $\varepsilon_0$ is the vacuum permittivity of free space.

The electromagnetic force between two point charges is proportional to the magnitude of the two charges and the inverse of the distance squared between them.

\textbf{Def:} One coulomb (C) is the magnitude of charge possessed by about $6 \cdot 10^8$ electrons.

\textbf{Principle of Superposition of Forces:} To find the total electromagnetic force acting on an object when there is more than one charge, just add the individual force vectors for each charge.



\end{document}