\documentclass{article}
\usepackage{tikz}
\usepackage{parskip}
\usepackage{xcolor}
\usepackage{tkz-euclide}
\usepackage[bottom=0.5in,top=0.5in,right=0.5in,left=0.5in]{geometry}
\usepackage{amsmath}
\usepackage{amsfonts}
\usepackage{amssymb}
\usepackage{amsthm}
\title{21.5: Electric Field Calculations}
\author{Alex L.}
\date{\today}
\pagecolor[rgb]{0,0,0} %black
\color[rgb]{1,1,1} %white

\begin{document}
\maketitle

\textbf{Exercise:} Charge $Q$ is uniformly distributed around a ring of radius $a$. Find the electric field at a point $P$ on the ring axis at a distance $x$ from the center.

\textbf{Solution:} We want to divide the ring into infintesimal segments $\delta S$. This segment only has contribution along the ring axis, as on the other side of the ring, there is an identical segment acting in the other direction transverse to the ring axis. The distance from the segment to the point charge is given by the equation $\sqrt{a^2 + x^2}$. As such, the contribution of $\delta S$ is  $\delta \vert \vec{E} \vert = \frac{\delta q}{4 \pi \varepsilon_0 (a^2 + x^2)}$. However, we have to take the only the component parallel to the axis, so we need to multiply the force vector by 

\end{document}