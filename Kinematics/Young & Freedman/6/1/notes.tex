\documentclass{article}
\usepackage{tikz}
\usepackage{parskip}
\usepackage{xcolor}
\usepackage{textcomp, gensymb}
\usepackage{pgfplots}
\usepackage{tkz-euclide}
\usepackage[bottom=0.5in,top=0.5in,right=0.5in,left=0.5in]{geometry}
\usepackage{amsmath}
\usepackage{amsfonts}
\usepackage{amssymb}
\usepackage{enumitem}
\usepackage{amsthm}
\pgfplotsset{compat=1.18}
\title{6.1: Work}
\author{Alex L.}
\date{\today}
\pagecolor[rgb]{0,0,0} %black
\color[rgb]{1,1,1} %white

\begin{document}
\maketitle

\textbf{Def:} \textbf{Work} is the change in kinetic energy of a system. It is defined by $W = \vec{F} \cdot \vec{d}$, the dot product of force and displacement.

Work is measured in joules (J), defined as Newtons $\cdot$ meters. 

Work can be positive if the force is acting in the direction of displacement, zero, if the force is perpendicular to displacement (carrying a briefcase), or negative, if force is opposite displacement (lowering a barbell).

Total work is the sum of all the work done by each individual force, or the work done by the net force. 



\end{document}