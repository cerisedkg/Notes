\documentclass{article}
\usepackage{tikz}
\usepackage{parskip}
\usepackage{xcolor}
\usepackage{textcomp, gensymb}
\usepackage{pgfplots}
\usepackage{tkz-euclide}
\usepackage[bottom=0.5in,top=0.5in,right=0.5in,left=0.5in]{geometry}
\usepackage{amsmath}
\usepackage{amsfonts}
\usepackage{amssymb}
\usepackage{enumitem}
\usepackage{amsthm}
\pgfplotsset{compat=1.18}
\title{6.2: Kinetic Energy and the Work-Energy Theorem}
\author{Alex L.}
\date{\today}
\pagecolor[rgb]{0,0,0} %black
\color[rgb]{1,1,1} %white

\begin{document}
\maketitle

\textbf{Def:} \textbf{Kinetic energy} is the energy a particle possesses due to its speed and mass, and is defined by $K = \frac{1}{2}mv^2$. It is a scalar quantity.

\textbf{Theorem:} \textbf{The Work-Energy Theorem} states that work is the change in a particle's kinetic energy, $W = K_1 - K_0 = \Delta K$. This means that kinetic energy is the total work doe to a particle to get it to its present speed from rest. 

\end{document}