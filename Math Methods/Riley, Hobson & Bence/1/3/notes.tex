\documentclass{article}
\usepackage{tikz}
\usepackage{parskip}
\usepackage{xcolor}
\usepackage{textcomp, gensymb}
\usepackage{pgfplots}
\usepackage{tkz-euclide}
\usepackage[bottom=0.5in,top=0.5in,right=0.5in,left=0.5in]{geometry}
\usepackage{amsmath}
\usepackage{amsfonts}
\usepackage{amssymb}
\usepackage{enumitem}
\usepackage{amsthm}
\pgfplotsset{compat=1.18}
\title{1.3: Coordinate Geometry}
\author{Alex L.}
\date{\today}
\pagecolor[rgb]{0,0,0} %black
\color[rgb]{1,1,1} %white

\begin{document}
\maketitle

\subsection{Lines}

The formula for a straight line graph is $$y = mx+c$$

An alternate form is $$ax + by + k = 0$$
where $m = \frac{-a}{b}$ and $c = \frac{-k}{b}$
and the x and y intercepts are $\frac{-k}{a}$ and $\frac{-k}{b}$ respectively.

If there is a power relationship between the two variables, $y = Ax^n$, then this can also be turned into a straight line by taking the natural log of both sides: $\ln y = n\ln x + \ln A$. In this form, the slope gives the power of $x$.

\subsection{Conics}

\textbf{Def:} A conic section takes the form $Ax^2 + By^2 + Cxy + Dx + Ey + F = 0$, and represent plane intersections of a double cone (an hourglass shape). They can take the form of either a parabola, a hyperbola, an ellipse, or a degenerate form, two straight lines. 

The standard form for an ellipse: $\frac{(x-a)^2}{a^2} + \frac{(y-b)^2}{b^2} = 1$

The standard form for a parabola: $(y-b)^2 = 4a(x-a)$

The standard form for a hyperbola: $\frac{(x-a)^2}{a^2} - \frac{(y-b)^2}{b^2} = 1$

If $C$ is nonzero, this indicates that the conic section is rotated, and so cannot be placed in standard form without a rotation of the axes.
\end{document}