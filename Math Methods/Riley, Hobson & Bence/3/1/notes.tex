\documentclass{article}
\usepackage{tikz}
\usepackage{parskip}
\usepackage{xcolor}
\usepackage{textcomp, gensymb}
\usepackage{pgfplots}
\usepackage{tkz-euclide}
\usepackage[bottom=0.5in,top=0.5in,right=0.5in,left=0.5in]{geometry}
\usepackage{amsmath}
\usepackage{amsfonts}
\usepackage{amssymb}
\usepackage{enumitem}
\usepackage{amsthm}
\pgfplotsset{compat=1.18}
\title{3.1: The Need for Complex Numbers}
\author{Alex L.}
\date{\today}
\pagecolor[rgb]{0,0,0} %black
\color[rgb]{1,1,1} %white

\begin{document}
\maketitle

\textbf{Def:} A \textbf{complex number} is a number in the form $a+bi$, where $i = \sqrt{-1}$. $a$ is called the \textbf{real term} of the complex number and $bi$ is called the \textbf{imaginary term}.

The conventional representation of a complex number is the letter $z = a+bi$. Sometimes, for compactness, a complex number is written $z = (a,b)$, with the real, then imaginary component in a tuple.

\textbf{Def:} An \textbf{Argand diagram} is a 2D plot of complex numbers, with the horizontal axis corresponding to the real component of a complex number, and the vertical axis corresponding to an imaginary component. 

\begin{tikzpicture}
    \draw [->] (-5,0) -- (5,0) node [above left] {$Re(z)$};
    \draw [->] (0,-5) -- (0,5) node [below right] {$Im(z)$};
    \filldraw[white] (3,2) circle (2pt) node [above right] {$z = a+bi$};
    \draw [dashed] (3,2) -- (3,0) node [below right] {$a$};
    \draw [dashed] (3,2) -- (0,2) node [above left] {$b$};
\end{tikzpicture}

\end{document}