\documentclass{article}
\usepackage{alexconfig}
\title{8.3: Matrices}

\begin{document}
\maketitle
A linear operator which transforms an $N$ dimensional vector space with basis $\mathbf{e}_j$ into an $M$ dimensional vector space with basis $\mathbf{f}_i$ can be represented by an $M$ by $N$ matrix ($M$ rows, $N$ columns).
$$
\mathcal{A}(\vec{r}) = \begin{bmatrix}
    A_{11} & A_{12} & A_{13} & ... & A_{1N}\\
    A_{21} & A_{22} & A_{23} & ... & A_{2N}\\
    A_{31} & A_{32} & A_{33} & ... & A_{3N}\\
    ... & ... & ...& ... & ...\\
    A_{M1} & A_{M2} & A_{M3} & ... & A_{MN}
\end{bmatrix}$$where $A_{ij}$ is the coefficient that transforms the $j$th basis vector in $\mathbf{e}_j$ into the $i$th basis vector component in $\mathbf{f}_i$.

If the dimensions the linear operator is transforming are the same, then the matrix is a square matrix. 

\begin{definition}[Vectors as Matrices]
We can write $N$ dimension vectors as $N$ by $1$ matrices, in terms of their components $x_i$ with respect to a basis $\mathbf{e}_i$. $$\vec{x} = \begin{bmatrix} x_1 \\ x_2 \\ x_3 \\ x_4 \\ ... \\ x_N\end{bmatrix}$$This type of matrix is called a \textbf{column matrix} or a \textbf{vector matrix}.

Alternatively, the vector can be written as a $1$ by $N$ transposed matrix:$$ \vec{x} = \begin{bmatrix}
    x_1 & x_2 & x_3 & x_4 & ... & x_N
\end{bmatrix}^T$$
\end{definition}
\end{document}