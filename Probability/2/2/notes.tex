\documentclass{article}
\usepackage{tikz}
\usepackage{parskip}
\usepackage{xcolor}
\usepackage{textcomp, gensymb}
\usepackage{pgfplots}
\usepackage{tkz-euclide}
\usepackage[bottom=0.5in,top=0.5in,right=0.5in,left=0.5in]{geometry}
\usepackage{amsmath}
\usepackage{amsfonts}
\usepackage{amssymb}
\usepackage{enumitem}
\usepackage{amsthm}
\pgfplotsset{compat=1.18}
\title{2.2: Sample Space and Events}
\author{Alex L.}
\date{\today}
\pagecolor[rgb]{0,0,0} %black
\color[rgb]{1,1,1} %white

\begin{document}
\maketitle

\textbf{Def:} The \textbf{sample space} $S$ is the set of all possible outcomes from an experiment. 

\textbf{Def:} Any subset $E \subset S$ is called an \textbf{event}. An event is a set consisting of some, but not all, possible outcomes of an experiment. If the outcome of an experiment is in $E$ we say that event has occured.

\subsection{De Morgan's Laws}

$$(\bigcup E_i)^c = (\bigcap E_i^c)$$

$$(\bigcap E_i)^c = (\bigcup E_i^c)$$

\end{document}