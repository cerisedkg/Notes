\documentclass{article}
\usepackage{tikz}
\usepackage{parskip}
\usepackage{xcolor}
\usepackage{textcomp, gensymb}
\usepackage{pgfplots}
\usepackage{tkz-euclide}
\usepackage[bottom=0.5in,top=0.5in,right=0.5in,left=0.5in]{geometry}
\usepackage{amsmath}
\usepackage{amsfonts}
\usepackage{amssymb}
\usepackage{enumitem}
\usepackage{amsthm}
\pgfplotsset{compat=1.18}
\title{2.3: Axioms of Probability}
\author{Alex L.}
\date{\today}
\pagecolor[rgb]{0,0,0} %black
\color[rgb]{1,1,1} %white

\begin{document}
\maketitle

\textbf{Def:} We define a \textbf{probability function} to be a function which maps a certain event $E$ in a sample space $S$ to a probability of that event happening, $P(E)$, and that obeys the following axioms:
\begin{enumerate}
    \item $0\leq P(E) \leq 1$
    \item $P(S) = 1$
    \item For any disjoint $E_0, E_1, E_2, ...$, $P(\cup E_i) = \sum P(E_i)$, that is, the probability of the union of disjoint events is the sum of the probablity of each individual event.
\end{enumerate}

These are the only three core assumptions that we must make about a probability function, everything else we know about probability can arise from these three axioms. 

\textbf{Deduction:} If we combine axioms $2$ and $3$, we get $P(S \cup \varnothing) = P(S) + P(\varnothing) = 1 + n$. Since the probability function can't output a result greater than one, $n$ must be zero, so we deduce that the probability of the empty set is equal to $0$.

\textbf{Deduction:} Notice that an event $E$ and its complement $E^C$ always partition the event space $S$, so $P(S) = P(E \cup E^C) = 1$, therefore, the probability of something occuring OR it not occuring is always $1$. 

By restructuring the above deduction, we get that $P(E^C) = 1- P(E)$.

\textbf{Theorem:} If $E \subset F$ then $P(E) \leq P(F)$. 

\textbf{Proof:} If $E \subset F$, then there must be some elements in $F$ that are not in $E$. Those elements will be in $E^C \cap F$. Then, $F = E + (E^C \cap F)$ so $P(F) = P(E) + P(E^C \cap F)$, and since $P(E^C \cap F) \geq 0$, then $P(F) \geq P(E)$

\end{document}