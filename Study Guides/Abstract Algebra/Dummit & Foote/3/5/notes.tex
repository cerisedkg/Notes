\documentclass{article}
\usepackage{alexconfig}
\title{3.5: Transpositions and Alternating Groups}
\author{Alex L.}
\date{\today}
\pagecolor[rgb]{0,0,0} %black
\color[rgb]{1,1,1} %white

\begin{document}
\maketitle

The elements of $S_n$ are the possible permutations of a set of size $n$. 

\textbf{Def:} Each element of $S_n$ can be written as a sequence of cycles of size 2, called \textbf{transpositions}. Imagine a cycle $(a_1,a_2,a_3,...,a_n)$. This moves element number $a_1$ to element number $a_2$, $a_2$ to $a_3$, and so on. We can describe this with the following sequence of transpositions (read right-to-left): $(a_1a_n)(a_1a_{n-1})...(a_1a_3)(a_1a_2)$. In this sequence, $a_1$ is used as a placeholder, and successive elements are moved with their previous element, in $a_1$. 

\textbf{Ex:} $\sigma = (1\ 12\ 8\ 10\ 4)(2\ 13)(5\ 11\ 7)(6\ 9)$ can be written as $(1\ 4)(1\ 10)(1\ 8)(1\ 12)(2\ 13)(5\ 7)(5\ 11)(6\ 9)$

\subsection{The Alternating Group}

Let $x_1, ... ,x_n$ be independent variables, and let $\Delta$ be a polynomial defined as $$\Delta = \prod_{1\leq i < j \leq n} (x_i - x_j)$$For example, when $n = 4$, we get $\Delta = (x_1 - x_2)(x_1 - x_3)(x_1-x_4)(x_2 - x_3)(x_2 - x_4)(x_3 - x_4)$. Notice how the second term's number is always greater than the first term's number. 

Now, lets define a function $\sigma(\Delta)$, which takes in a delta function and permutes each number according to an element of $S_n$. For example, if we chose $(1,2,3,4)$ and our polynomial from above, we get $\sigma(\Delta) = (x_2 - x_3)(x_2 - x_4)(x_2 - x_1)(x_3 - x_4)(x_3-x_1)(x_4 - x_1)$, as you can see, the number of every variable gets mapped to a new number according to our permutation.

However, now some of the first terms are larger than the second terms, for example, in our example, $(x_2 - x_1), (x_3 - x_1), (x_4-x_1)$ are now all out of order. To fix this, we can factor out a minus sign and get $-(x_1 - x_2), -(x_1 - x_3), - (x_1 - x_4)$. We then multiply out these minus signs and get that the overall sign of $\sigma(\Delta)$ is now $-1$, but the individual terms haven't changed. 

As a matter of fact, no matter what permutation you choose for $\sigma$, the result of $\sigma(\Delta) = \pm \Delta$.

For each $\sigma$ in $S_n$, lets define a function that tells us if $\sigma(\Delta)$ is positive or negative. We will call this function $\epsilon(\sigma)$, or the sign function.

\textbf{Def:} 
\begin{enumerate}
    \item $\epsilon(\sigma)$ is called the sign of $\sigma$
    \item $\sigma$ is called an even permutation if $\epsilon(\sigma) = 1$ and an odd permutation if $\epsilon(\sigma) = -1$
\end{enumerate}

\textbf{Prop:} The map $\epsilon: S_n \rightarrow \{1,-1\}$ is a homomorphism. 

\textbf{Proof:} Let $\tau, \sigma$ be elements of $S_n$. Then, $\epsilon(\tau\sigma) = \tau \cdot \sigma (\Delta) = \prod_{1\leq i < j \leq n} (x_{\tau\sigma(i)} - x_{\tau\sigma(j)})$. Lets evaluate just $\sigma(\Delta)$ first, and this will result in $\Delta$, but with $k$ factors of the form $(x_j - x_i)$, which we will flip. The end result is $\epsilon(\sigma) \Delta$, with all the values of $\Delta$ in order as they were originally. Then, we evaluate $\tau$, and order the variables again, to get $\epsilon(\tau)\epsilon(\sigma) \Delta$, showing that $\epsilon$ is a homomorphism. 

\textbf{Prop:} Transpositions are odd permutations and $\epsilon$ is a surjective homomorphism.

\begin{customproof}

Let $n = 4$, $\sigma = (1234)$, $\tau = (432)$ and so $\tau\sigma = (1324)$. 

Then, $(\tau\sigma)(\Delta) = (1324)\Delta = (x_3 - x_4)(x_3 - x_2)(x_3 - x_1)(x_4-x_2)(x_4-x_1)(x_2-x_1) = (-1)^5\Delta$

On the other hand, $\tau(\sigma\Delta) = (x_{\tau(2)} - x_{\tau(3)})(x_{\tau(2)} - x_{\tau(4)})(x_{\tau(2)} - x_{\tau(1)})(x_{\tau(3)} - x_{\tau(4)})(x_{\tau(3)} - x_{\tau(1)})(x_{\tau(4)} - x_{\tau(1)}) = (x_3 - x_4)(x_3 - x_2)(x_3 - x_1)(x_4 - x_2)(x_4 - x_1)(x_2 - x_1) = (-1)^{5}(\Delta)$

Now, lets take the case of the transpose $(1,2)$. $\epsilon (1,2) = -1$, because it flips only one factor, $(x_1 - x_2)$. Now note that $(i,j) = \lambda(1,2)\lambda$. Since we have two lambdas, their sign cancels, and we just have $\epsilon(i,j) = \epsilon (1,2) = -1$.
\end{customproof}

\begin{definition}[Alternating Groups]
The alternating group $A_n$ is the kernel of the homomorphism $\epsilon$, or the set of all even permutations. The order of $A_n = \frac{1}{2} \vert S_n\vert = \frac{1}{2}n!$
\end{definition}

\begin{proposition}
A permutation $\sigma$ is odd if and only if the number of cycles of even length in its decomposition is odd
\end{proposition}

\begin{customproof}
An odd cycle times an odd cycle is an even cycle, and an even cycle times an odd cycle is odd. A cycle of size $m$ decomposes into $m-1$ transposes, which are odd, so an even length cycle has an odd parity. An even number of even cycles results in an even permutation, so the number of even cycles must be odd for the permutation to be odd. 
\end{customproof}

\end{document}