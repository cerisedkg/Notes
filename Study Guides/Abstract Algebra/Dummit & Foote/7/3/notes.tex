\documentclass{article}
\usepackage{alexconfig}
\title{7.3: Ring Homomorphisms}

\begin{document}
\maketitle

\begin{definition}[Ring Homomorphism]
Let $R$ and $S$ be rings. Then, a mapping $\varphi: R \to S$ is a \textbf{ring homomorphism} if:

\begin{enumerate}
    \item $\varphi(a+b) = \varphi(a) + \varphi(b)$
    \item $\varphi(ab) = \varphi(a)\varphi(b)$
\end{enumerate}

The kernel $\text{ker } \varphi$ is the set of all elements in $R$ which map to $0$ in $S$. 

A bijective ring homomorphism is called a ring isomorphism.
\end{definition}

\begin{proposition}
Let $R$ and $S$ be rings and let $\varphi: R \to S$ be a ring homomorphism. Then:
\begin{enumerate}
    \item The image of $R$ is a subring of $S$
    \item The kernel of $\varphi$ is a subring of $R$
\end{enumerate}
\end{proposition}

\begin{customproof}
\begin{enumerate}
    \item Elements of the image of $R$ take the form $\varphi(a)$, where $a$ is in $R$. The image of $R$ is commutative because $\varphi(a) + \varphi(b) = \varphi(a+b) = \varphi (b+a) = \varphi(b)+\varphi(a)$. 
\end{enumerate}
\end{customproof}

\end{document}