\documentclass{article}
\usepackage{alexconfig}
\title{3.5: Two Dimensional Systems and their Vector Fields}

\begin{document}
\maketitle
\section{Motivation}
We want to explore how vector fields of autonomous systems look.

\section{Content}
Suppose we have a constant coefficients autonomous system $$\begin{bmatrix}
    \frac{dx}{dt} \\[3pt] \frac{dy}{dt}
\end{bmatrix}= \begin{bmatrix}
    a &\: b \\[3pt] c &\: d
\end{bmatrix}\begin{bmatrix}
    x \\[3pt] y\\
\end{bmatrix}$$

Solutions to this system look like $$\begin{bmatrix}
    x \\y
\end{bmatrix} = c_1\vec{v_1}e^{\lambda_1 t} + c_2\vec{v_2}e^{\lambda_2 t}$$where $\vec{v}$ and $\lambda$ are eigenvectors and eigenvalues respectively.

Eigenvalues will determine how a phase diagram looks. Lets start by plugging in an eigenvector $\alpha \vec{v}$. We get $$\begin{bmatrix}
    x' \\ y'
\end{bmatrix} = \alpha\lambda \vec{v}$$because $P\vec{v} = \lambda\vec{v}$. 

\begin{proposition}
    \begin{enumerate}
        \item If both of our eigenvalues are real and positive, we can see that the derivative of an eigenvector always points away from the origin. There is a critical point called a source at the origin.

\item If both of our eigenvalues are real and negative, the derivative at an eigenvector points toward the origin. This is because, whatever the sign of $\alpha \vec{v}$ is, the sign of $\alpha\lambda\vec{v}$ is the opposite, which takes you closer to the origin. There is a critical point called a sink at the origin

\item If one eigenvalue is positive and one eigenvalue is negative, then the directionality of the derivative depends on where you start, and we say there is a saddle critical point at the origin.

\item If both eigenvalues are purely imaginary, with no real component, then the vector field forms an ellipse, and you orbit the origin, which is a center critical point.
\item If the eigenvalues are complex with a positive real part, then the derivative spirals away from the origin, and we get a spiral source critical point
\item If both eigenvalues are complex with an imaginary real part, then the derivative spirals towards the origin, and we get a spiral sink critical point.
    \end{enumerate}

\end{proposition}

\begin{customproof}
\begin{enumerate}
    \item Since $\lambda$ is positive, the sign of $\alpha\vec{v}$ is equal to the sign of $\lambda\alpha\vec{v}$, so they point in the same direction. Since the eigenvector always points away from the origin, so will the derivative.
    \item Since $\lambda$ is negative, the sign of $\alpha\vec{v}$ is opposite the sign of $\lambda\alpha\vec{v}$, so they point in opposite directions. Since the eigenvector always points away from the origin, the derivative always points towards the origin.
    \item This is a mixture of the previous two, so depending on which angle you approach the origin from, the derivative will point in a different direction.
    \item With only imaginary components, our solution looks something like $$y=\begin{bmatrix}
        c_1\cos(bx) \\ c_2\sin(bx)
    \end{bmatrix} + \begin{bmatrix}
        c_1\sin{bx} \\ -c_2\cos{bx}
    \end{bmatrix}$$which is the parametric equation for an ellipse, so our solution is elliptical.
    \item If we have a positive real part, then our solutions look like $$y=e^t\begin{bmatrix}
        c_1\cos(bx) \\ c_2\sin(bx)
    \end{bmatrix} + e^t\begin{bmatrix}
        c_1\sin{bx} \\ -c_2\cos{bx}
    \end{bmatrix}$$and since $e^t$ is constantly growing, our solutions grow in magnitude as well.
    \item If we have a negative real part, then our solutions look like $$y=e^{-t}\begin{bmatrix}
        c_1\cos(bx) \\ c_2\sin(bx)
    \end{bmatrix} + e^{-t}\begin{bmatrix}
        c_1\sin{bx} \\ -c_2\cos{bx}
    \end{bmatrix}$$and since $e^{-t}$ is constantly shrinking, our solutions shrink in magnitude as well.
\end{enumerate}
\end{customproof}
\end{document}