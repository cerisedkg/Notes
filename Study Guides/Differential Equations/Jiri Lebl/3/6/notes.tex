\documentclass{article}
\usepackage{alexconfig}
\title{3.6: Second Order Systems and Applications}

\begin{document}
\maketitle
\section{Motivation:}
In this section, we want to learn how to solve more complex systems of springs and masses.

\section{Content}
\begin{example}
Suppose we have a system of three masses, $m_1$, $m_2$, and $m_3$, and a spring with constant $k_1$ connects $m_1$ to the wall, $k_2$ connects $m_1$ and $m_2$, $k_3$ connects $m_2$ and $m_3$, and $k_4$ connects $m_3$ to the other wall. Assume there are no forces other than the spring forces on the masses. Can you find equations governing the motion of the three masses?
\end{example}

\begin{solution}
Let $x_1$, $x_2$, $x_3$ be the displacement from the rest positions of each mass, with right being positive. Now we write the force equations for each mass:

For $m_1$, only $k_1$ and $k_2$ are acting on it, so the equation is $$m_1 x''_1 = -k_1x_1 + k_2(x_2-x_1)$$Since the force of a spring acts in the opposite direction of displacement, $k_1x_1$ is negative and $k_2(x_2-x_1)$ is positive. 

For $m_2$, $k_2$ and $k_3$ act on it, so $$m_2 x''_2 = -k_2(x_2-x_1) + k_3(x_3-x_2)$$

For $m_3$, $k_3$ and $k_4$ act on it, so $$m_3x''_3 = -k_3(x_3-x_2) - k_4x_3$$

Rearranging all three equations, we get:

$$m_1x''_1 = -(k_1+k_2)x_1 + k_2x_2$$$$m_2x''_2 = k_2x_1 -(k_2 + k_3)x_2 + k_3x_3$$$$m_3x''_3 = k_3x_2 - (k_3 + k_4)x_3$$

Putting these in matrix form, we get $$M\vec{x}'' = K\vec{x}$$where $$M = \begin{bmatrix}
    m_1 & 0 & 0\\
    0 & m_2 & 0\\
    0 & 0 & m_3\\
\end{bmatrix}$$$$K = \begin{bmatrix}
    -(k_1+k_2) & k_2 & 0\\
    k_2 & -(k_2+k_3) & k_3\\
    0 & k_3 & -(k_3+k_4)\\
\end{bmatrix}$$Now, we want to isolate $\vec{x}''$, so we left multiply by $$M ^{-1} = \begin{bmatrix}
    \frac{1}{m_1} & 0 & 0\\
    0 & \frac{1}{m_2} & 0\\
    0 & 0 & \frac{1}{m_3}\\
\end{bmatrix}$$ to get $$\vec{x}'' = M ^{-1} K \vec{x}$$Multiplying out the matrices, we get $$A = \begin{bmatrix}
    \frac{-(k_1+k_2)}{m_1} & \frac{k_2}{m_1} & 0\\
    \frac{k_2}{m_2} & \frac{-(k_2 + k_3)}{m_2} & \frac{k_3}{m_2}\\
    0 & \frac{k_3}{m_3} & \frac{-(k_3+k_4)}{m_3}
\end{bmatrix}$$The actual contents of the matrix are unimportant, but now our equation looks like $$\vec{x}'' = A\vec{x}$$Notice how all of the values in the coefficient matrix are constants, so we can solve this using the constant coefficients eigenvalue method.

Lets guess that the vector of our solutions looks like $\vec{x} = \vec{v}e^{\alpha t}$ where $\vec{v}$ is an eigenvector. Then, $\vec{x}'' = \alpha^2e^{\alpha t}$. Plugging into our differential equation, we get that$$\alpha^2e^{\alpha t} = A\vec{v}e^{\alpha t}$$We then divide by $e^{\alpha t}$ to get$$\alpha^2\vec{v} = A\vec{v}$$Therefore, $\alpha^2$ must be an eigenvalue of $A$ and $\vec{v}$ must be an eigenvector of $A$. 

Now, usually, $\alpha^2$ will be negative, so $\alpha = \pm i\omega$, a complex number. Then, plugging back into our guess, we get that $$\vec{x} = \vec{v}e^{\alpha t} = \vec{v}(\cos(\omega t) + i\sin(\omega t))$$By taking the real and imaginary components, we get two solutions: $\vec{v}\cos(\omega t)$ and $\vec{v}\sin(\omega t)$
\end{solution}

This solution works for any system of $n$ masses and $n+1$ springs! You just make an $n$-dimensional matrix, and everything works!

\begin{example}
Suppose we have two masses, $m_1$ and $m_2$, with a spring connecting $m_1$ to the left wall, and a spring connecting $m_2$ to $m_1$. Given that $m_1 = 2, m_2 = 1, k_1 = 4, k_2 = 2$, wherehat are the equations of motion for this system?
\end{example}

\begin{solution}
Lets begin by defining $x_1$ to be the displacement from when $m_1$ is at rest (right is positive), and $x_2$ to be the displamenent from when $m_2$ is at rest.

Lets start by writing our force equations: $$m_1x''_1 = -k_1x_1 + k_2(x_2-x_1)$$$$m_2x''_2 = -k_2(x_2-x_1)$$Rearranging, we get $$m_1x''_1 = -(k_1+k_2)x_1 + k_2x_2$$ $$m_2x''_2 = k_2x_1-k_2x_2$$Putting it in matrix form, we get $$M\vec{x}'' = K\vec{x}$$where$$M = \begin{bmatrix}
    m_1 & 0 \\ 0 & m_2
\end{bmatrix}$$$$K = \begin{bmatrix}
    -(k_1+k_2) & k_2\\k_2 & -k_2
\end{bmatrix}$$Now, we left-multiply by $M ^{-1}$ to get $$\vec{x}'' = A\vec{x}$$where$$A = \begin{bmatrix}
    \frac{-(k_1+k_2)}{m_1} & \frac{k_2}{m_1} \\ \frac{k_2}{m_2} & \frac{-k_2}{m_2}
\end{bmatrix}$$Plugging in, we get $$A = \begin{bmatrix}
    -3 & 1 \\ 2 & -2
\end{bmatrix}$$The eigenvalues of this matrix are $\lambda = -1, -4$, and the corresponding eigenvectors are $\begin{bmatrix}
    1\\2
\end{bmatrix}, \begin{bmatrix}
    1 \\ -1
\end{bmatrix}$. We can calculate that $\omega = 1, 2$, and our solutions are $$\vec{x} = \begin{bmatrix}
    1\\2
\end{bmatrix}(a_1\cos(t) + b_1\sin(t)) + \begin{bmatrix}
    1\\-1
\end{bmatrix}(a_2\cos(2t) + b_2\sin(2t))$$Our two natural frequencies of oscillation are $1$ and $2$, and the two terms in the solution are called normal modes. 
\end{solution}

%Add a section about forced oscillations.
\subsection{Forced Oscillation}

Suppose our system has a driving displacement, so that our equation is $$\vec{x}'' = A\vec{x} + \vec{F}\cos(\omega t)$$

Remember that to form a solution to a non-homogeneous constant coefficient equation, we need a complementary solution and a particular solution. We'll just solve the particular solution right now.

Lets guess that our particular solution $\vec{x}_p = \vec{c}\cos(\omega t)$. We don't include a $\sin$ term because we only have a second derivative.

We calculate that $\vec{x}''_p = -\omega^2\vec{c}\cos(\omega t)$, and plugging in, we get $$-\omega^2\vec{c}\cos(\omega t) =A\vec{c}\cos(\omega t) + \vec{F}\cos(\omega t)$$Rearranging, we get $$(A + \omega^2I)\vec{c} = - \vec{F}$$therefore$$\vec{c} = (A+\omega I) ^{-1} (-\vec{F})$$
\end{document}