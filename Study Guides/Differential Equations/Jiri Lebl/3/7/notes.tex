\documentclass{article}
\usepackage{alexconfig}
\title{3.7: Multiple Eigenvalues}

\begin{document}
\maketitle

\section{Motivation}
What do we do when we encounter repeated eigenvalues and defective eigenvalues?

\section{Content}

\subsection{Repeated Eigenvalues}

\begin{definition}[Repeated Eigenvalues]
Whenever there is a repeated root of the characteristic equation $$\text{det}(A-\lambda I) = 0$$with multiplicity $m$, then we have a repeated eigenvalue. $m$ is called the \textbf{algebraic multiplicity}. The number of linearly independent eigenvectors corresponding to this eigenvalue is a distinct number, called the \textbf{geometric multiplicity}. The geometric multiplicity is also the dimension of the corresponding \textbf{eigenspace} (the span of all eigenvectors associated with a corresponding eigenvalue).  
\end{definition}
The above definition makes sense if you think about it. If we have an eigenvalue with two associated eigenvectors, $\vec{v}_1$ and $\vec{v}_2$, then linear combinations of them also result in eigenvectors, because $$A(\vec{v}_1 + \vec{v}_2) = A\vec{v}_1 + A\vec{v}_2 \text{ (by properties of matrix multiplication)} = \lambda\vec{v}_1 + \lambda\vec{v}_2 = \lambda(\vec{v}_1 + \vec{v}_2)$$And so having multiple linearly independent eigenvectors means you get an eigenspace of that many dimensions.

If we get a matrix with repeated eigenvalues, just assign a unique eigenvector to each term, so for a matrix $$A = \begin{bmatrix}
    3 & 0\\ 0 & 3
\end{bmatrix}$$which has eigenvalues $\lambda = 3$ and eigenvectors $\begin{bmatrix}
    1\\0
\end{bmatrix}$,$\begin{bmatrix}
    0\\1
\end{bmatrix}$ our result looks like $$\vec{x} = c_1\begin{bmatrix}
    1\\0
\end{bmatrix}e^{3t} + c_2 \begin{bmatrix}
    0\\1
\end{bmatrix}e^{3t}$$

\subsection{Defective Eigenvalues}
\ 
\begin{definition}
Sometimes, the eigenvectors for our repeated eigenvalue are not linearly independent. In this case, our geometric multiplicity is less than our algebraic multiplicity, and we call the eigenvalue \textbf{defective}. The difference between the two is called the \textbf{defect}.   
\end{definition}

Suppose we have two linearly dependent eigenvalues $\vec{v}_1$ and $\vec{v}_2$. We can't form a general solution out of these, so lets just slap an $x$ on one of them (in this case, its a $t$ as our independent variable but I digress). 

We get $$\vec{x} = \vec{v}_1 e^{\lambda_1t} + t\vec{v}_2 e^{\lambda_1t}$$Lets see if this is actually a solution. Differentiating, we get $$\vec{x}' = (\lambda_1\vec{v}_1 + \vec{v}_2)e^{\lambda_1 t} + \lambda_1t\vec{v}_2e^{\lambda_1 t}$$Our solution must satisfy $A\vec{x} = \vec{x}'$ so plugging in our original equation, we get $$\vec{x}' = A\vec{v}_1e^{\lambda_1t} + At\vec{v}_2e^{\lambda_1t}$$Matching terms and then rearranging, we get that $$(A-\lambda I)\vec{v}_1 = \vec{v}_2$$and$$(A-3I)\vec{v}_2 = 0$$and so our new vector has to satisfy these two equations. 
\end{document}