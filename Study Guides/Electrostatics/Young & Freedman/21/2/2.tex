\documentclass{article}
\usepackage{tikz}
\usepackage{parskip}
\usepackage{xcolor}
\usepackage{tkz-euclide}
\usepackage[bottom=0.5in,top=0.5in,right=0.5in,left=0.5in]{geometry}
\usepackage{amsmath}
\usepackage{amsfonts}
\usepackage{amssymb}
\usepackage{amsthm}
\title{21.2: Conductors and Insulators}
\author{Alex L.}
\date{\today}
\pagecolor[rgb]{0,0,0} %black
\color[rgb]{1,1,1} %white

\begin{document}
\maketitle
\textbf{Def:} \textbf{Conductors} are materials that allow charges to flow freely throughout the material.

\textbf{Def:} \textbf{Insulators} are materials that do not allow charges to flow freely.

Most metals are conductors, and most nonmetals are insulators.

\textbf{Induced Charges} By bringing a positively charged object near the conductor, electrons are attracted near the charged object. As such, the conductor will develop a slight negative charge on the side near the object, and a positive charge on the side away from the object, caused by the electrons being drawn away from that area. Likewise, a negatively charged object will induce a positive charge on the side nearest to it by repelling nearby electrons.

This effect can also happen in insulators, although to a much lesser degree, which is why you can pick up paper scraps with a charged comb. 
\end{document}