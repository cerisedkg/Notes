\documentclass{article}
\usepackage{tikz}
\usepackage{parskip}
\usepackage{xcolor}
\usepackage{tkz-euclide}
\usepackage[bottom=0.5in,top=0.5in,right=0.5in,left=0.5in]{geometry}
\usepackage{amsmath}
\usepackage{amsfonts}
\usepackage{amssymb}
\usepackage{amsthm}
\title{21.4: Electric Field and Electric Forces}
\author{Alex L.}
\date{\today}
\pagecolor[rgb]{0,0,0} %black
\color[rgb]{1,1,1} %white

\begin{document}
\maketitle

\textbf{Def:} A charged object produces an electric field around itself. The electric field is a property of space that mediates the electromagentic force, and it is a vector field.

\textbf{Def:}  To find what an electric field looks like at a point, we place a \textbf{test charge} within the field, which is an imaginary point with negligible mass, and a known charge.

The force exerted on a charged object $a$ by an electric field is given by the equation: $$\vec{F_a} = q_a\vec{E}$$where $q$ is the charge of the object and $\vec{E}$ is the electric field vector at the location. The unit of electric field magnitude is $N/C$ (newtons per coulomb).

In the case of a test charge, y substituting the force vector with Coulomb's Law, we get $\frac{1}{4 \pi \varepsilon_0}\frac{\vert q\cdot q_0\vert}{r^2} = q_0 \vec{E}$, and simplifying, we get $$\vert\vec{E}\vert = \frac{q}{4 \pi \varepsilon_0 r^2}$$and we can multiply it by a unit vector pointing from the point charge to the test charge, $\hat{r}$, to get the direction of the field vector as well.



\end{document}