\documentclass{article}
\usepackage{alexconfig}
\title{21.7: Electric Dipoles}

\begin{document}
\maketitle

\begin{definition}[Electric Dipoles]
    \textbf{Electric dipoles} are pairs of opposite point charges separated by a distance $d$.
\end{definition}

When placed in an electric field, dipoles will not move, but experience a torque as they try to align with the field. 

The torque is calculated with respect to the center of the dipole. If $\phi$ is the angle between the electric field and the dipole's axis, then the force acting on one charge is $\frac{qEd\sin\phi}{2}$, but since there are two charges, with torques acting in the same direction, the total torque on the dipole is $$qEd\sin\phi$$

\begin{definition}[Dipole Moment]
We define the \textbf{electric dipole moment}, $\vec{p}$, as $p = q\vec{d}$, or the magnitude of the charges times the displacement between them. The unit for the dipole moment is Coulomb-meters.  
\end{definition}

We can now define $\tau = \vert \vec{p}\vert E\sin\phi$, or alternatively, $$\tau = \vec{p}\times \vec{E}$$

\subsection{Potential Energy of a Dipole}
The work done by torque is negative, because the angle $\phi$ is decreasing.
$$dW = \tau d\phi = -pE\sin\phi d\phi$$and as such, $$W = pE\cos (\Delta\phi) = -U$$

We can rewrite this as a dot product: $$U = -\vec{p} \cdot \vec{E}$$

If a dipole isn't uniform, the net force on the dipole in a field may be nonzero. 
\end{document}