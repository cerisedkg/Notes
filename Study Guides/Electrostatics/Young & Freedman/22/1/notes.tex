\documentclass{article}
\usepackage{alexconfig}
\title{22.1: Charge and Electric Flux}

\begin{document}
\maketitle

\begin{definition}[Gauss's Law]
\textbf{Gauss's Law} states that there is a relationship between the charges enclosed by an imaginary surface, and the net strength of the electric field on the imaginary surface, called \textbf{electric flux}. Elecric flux is a measure of the total strength of an electric field going through a surface, just like how fluid flow is the total volume of fluid passing through a surface at any given time.
\end{definition}

\begin{example}
Imagine we drew an imaginary box around a negative point charge. At every point on the box's surface, the electric field is pointing inwards, and so a negative charge implies an inward flux, and vice versa for positive charges.

If our surface encloses both a negative and positive charge, then half of the surface will have inward flux, and half the surface will have an outward flux, therefore, the net flux will be zero.

If our surface encloses an empty region of space, but outside of the surface, there is a strong electric field, then the net flux will be zero, as there is exactly as much flux going in as going out. 
\end{example}

\end{document}