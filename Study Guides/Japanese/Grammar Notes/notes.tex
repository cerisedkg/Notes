\documentclass{article}
\usepackage{alexconfig}
\title{Japanese Grammar Notes}
\usepackage{CJKutf8}


\begin{document}
\maketitle

\begin{CJK}{UTF8}{min}

\section{Motivation}
This is an explainer document on all of the properties of Japanese grammar I have encountered.
\section{Sentence Structure}
Basic Japanese sentence structure goes subject, nouns/objects, and a final verb, is generally more freeform than English sentence structure, and that is because Japanese uses markers called particles attached to certain clauses.

\begin{definition}[Particles]
Particles, or 助詞(じょし), are markers placed after nouns, adjectives, and verbs, that modify their meaning, or clarify their usage.
\end{definition}

\begin{example}
メアリさん\textbf{は}図書館\textbf{で}二時\textbf{に}勉強します
\end{example}
As we can see above, the highlighted kana (which means hiragana \& katakana) above are a few examples of the many many particles in Japanese, and we will go into more detail in the particles section. 

Japanese sentences have a train-car structure, where clauses and even verb conjugations can "link" one after another, and the language overall is very modular. 

Not every clause needs to have a particle, however. Propositional phrases like "みぎ" (in front of), or like "二時間" (two hours in duration), can stand independently, and these are called nude nouns.

In addition, while English is a stickler for fully formed thoughts in sentences, Japanese is much more lax. For example, take this sample conversation:

\begin{example}
たけし:メアーリさんはカフェでコーヒーをのみましたか?
ロバト:いいえ、お茶をのみました。
\end{example}

In the example above, we see that since Takeshi-san is starting the conversation, he provides much of the needed context, like where and what he is inquiring about, but Robert-san replies with a very stripped-down reply. Directly translated, we have:

\begin{example}
Takeshi-san: Did Mary-san drink coffee at the cafe?
Robert-san: No, drank tea.
\end{example}

As opposed to saying "No, Mary drank tea at the cafe", which would be an appropriate response in English, instead we strip all of the fluff out. Japanese conversations resemble lego blocks, each person building off ideas from the last, not presenting already wholly formed ideas like in English. 

\section{Particles}
We detail some common particles and their usages below:

\subsection{は}
は when used as a particle is pronounced "wa", and is the topic marker. It sets the topic for a sentence or conversation, but is not always the subject of the sentence. 

\begin{example}
メーアリさんの選考\textbf{は}経済です。
\end{example}

In this sentence, we are talking about Mary-san's major, so we mark that with the は particle.

\begin{example}
    テレビ\textbf{は}子供が見ます.
\end{example}
This sentence translated is "The child watches the television", but as we can see, the topic is not the subject. The subject is the child, but we are talking about the television, so it is marked as the topic. 

\subsection{が}

The particle "が" is the subject marker, and marks the subject of a verb, and can also mark the topic of a sentence. The difference between "が" and "は" is that "が" shifts more emphasis to the noun preceeding it while "は" emphasizes the verb more.

\begin{example}
教科書があります
\end{example}

The above sentence translates to "There exists a textbook" or "I have a textbook" (context dependent), and as we can see, it is used with "が" because the thing that is existing is quite important to the sentence.

\begin{example}
私の名前は梁光です
\end{example}

In contrast, when we say that something is something else using "です", we use "は" because, like in the above example, the topic (my name) plays a comparatively less important role, as its being equated to something else.

\subsection{を}

The particle "を" is a direct object marker, and is used for direct objects.

\begin{example}
窓を閉めてください
\end{example}

The above sentence translates to "Please close the window", and as we see, the "を" particle marks the thing we are doing our action to, the direct object. 

\subsection{に}

The particle "に" plays a variety of roles, but most notably, it marks:

\begin{enumerate}
    \item The goal of a verb
    \item When describing the absolute time when something happens
\end{enumerate}

\begin{example}
図書館に行きますか
\end{example}

The above section translates to "Are you going to the library?", and as we can see, the library is the goal of the verb, so we use the "に" particle.

\begin{example}
私に工学を教えてください
\end{example}

The above example translates to "Please teach engineering to me". As we can see engineering is the direct object so we mark it with "を" and the goal is to teach it to me, so "私" is marked with a "に".

\begin{example}
昨日ばんごはんを六時半ごろに食べました
\end{example}

This translates to "Yesterday I ate dinner at around 6:30". As we can see, "昨日", the relative time term, does not get a "に" particle, it is naked, but "六時半ごろ", or "around 6:30" does acutally get a "に" par   ticle, as it is a measure of absolute time. 

Words like "週末" and "朝","昼", "晩" can get a "に" particle depending on personal perference. 

%cover de for location, no for linking nouns or turning verbs into nouns, mo for saying something is also something (cover when mo can replace and where it has to be tacked onto the ends)

\section{Verbs and Conjugation}

%Conjugations, tenses, formal and informal, ichidan and godan verbs, te form

In Japanese, there are two classes of verbs with different conjugations - ichidan verbs and godan verbs, as well as three irregular verbs, する, to do, くる, to come, and the copular verb です, to be. 

Japanese verbs conjugate like train cars, and by swapping out different ending vowels, you can allow verbs to accept different helper verbs and adjectives on the end, changing their meaning.

All verbs by default start in the infinitive form ending with an "う" (dictionary form), but godan can be conjugated to every other vowel, "あ” (generally for negative forms), "え" (generally for imperative forms), "い" (generally for polite forms), and "お" (generally for volitional forms), while 一段 verbs conjugate just by dropping the final "る".

\subsection{一段 Verbs}

Ichidan verbs are far less common than godan verbs, and end in "-eru" and "-iru" (though not all verbs ending in "-eru" or "-iru" are 一段). When conjugating ichidan verbs, you drop the ending "る". The stem can now accept any helper adjectives or verbs

\begin{example}
\ 

食べる $\rightarrow$ 食べ-

開ける $\rightarrow$ 開-

起きる $\rightarrow$ 起き-
\end{example}

\subsection{五段 Verbs}

Godan verbs are more common, and use stems like before. Which stem you use depends on which helper verbs/adjectives you attach.

To change stems, simply swap the last vowel.

\begin{example}
\ 

    Turning "-う” stem to "-い” stem:

飲む $\rightarrow$ 飲み-

行く $\rightarrow$ 行き-

もつ $\rightarrow$ もち-
\end{example}

\subsection{Irregular Verbs: する and くる}

These two verbs conjugate like 一段 verbs, but we also modify the character before them.

\begin{example}
    \ 

する $\rightarrow$ し-

くる $\rightarrow$ き-
\end{example}

\subsection{Special Case: です}

です is a special copular (meaning to express what things are) verb used to turn non-predicative (sentence ending) phrases into predicative ones. Parts of speech like nouns, and な-adjectives are unable to end sentences by themselves, so a sentence like:

\begin{example}
私の名前はアレクス
\end{example}
is unnatural because you can't end a sentence on the noun "アレクス”. This would be equvalent to saying "My name Alex" in English. As such, we add the verb "です", meaning "to be", capping off our sentence, and completing it to form

\begin{example}
私の名前はアレクスです
\end{example}

or "My name is Alex". Its important to note that い-adjectives are predicative in Japanese, but that'll be covered more in the adjectives section.

%Add conjugations for desu

\begin{example}
\ 



\begin{tabular}{| c | c || c | c |}
    です conjugations: & & Positive & Negative\\\hline\hline
    Present: & Colloquial & だ & でわない\\
    & Polite & です & でわありません\\
    Past: & Colloquial & だった & でわなっかた\\
    & Polite & でした & 出羽ありませんでした\\
    Probable: & Colloquial & だろ & でわないだろ\\
    & Polite & でしょ & でわないでしょ\\
\end{tabular}
\end{example}

%maybe add a special case for aru? negative form conjugates differently.

\subsection{-ます stem}

If you put a verb into -ます form and then chop of the -ます, what you are left with is called the -ます stem. 

For 一段 verbs, all you need to do is attach endings to the -ます stem to conjugate.

For ご段 verbs, you need to change the last vowel of the -ます stem to accept whatever helper verb/adjective you are adding. 

\section{Helper Verbs and Adjectives}

\subsection{Dictionary Form}

\subsubsection{Usage}

The dictionary form of a verb is the default form of every Japanese verb. 

It is informal and can be used to talk about what you are doing right now, what you regularly do as a habit, and what you will do in the future.

\subsubsection{Conjugation}

Just keep the verb in its default packaging.

\begin{example}
    \ 

うん、寿司を\underline{食べる}

毎種末\underline{買い物す}

明日\underline{勉強する}
\end{example}
\begin{solution}
    \ 

Yes, I eat sushi.

Every weekend, I go shopping.

I will study tomorrow.
\end{solution}

\subsection{-ます: Formal Dictionary Form}

\subsubsection{Usage}

The polite dictionary form functions like the dictionary form, but is used in polite settings.

\subsubsection{Conjugation}

For 一段 verbs, simply drop the -る and add the -ます in its place. For 五段 verbs, modify the ending vowel by dropping the -う and turning it into -い.

\begin{example}
\ 

寿司を食べます

毎日, 勉強します

もう一度聞きます
\end{example}

\begin{solution}
\ 

I eat sushi

Every day, I study

I will ask one more time
\end{solution}

The irregular verbs する and くる conjugate to します and きます respectively.

\subsection{-て: Informal Te-form}

The -て form is an extremely versatile verb conjugation that can act as a verb ending but also accept other phrases as well.

Although technically it is an informal form, it is perfectly fine to use in polite settings. 

\subsubsection{Usage}

The -て form by itself is an informal request to do something.

To make the request more polite, add "ください", meaning "please" (a modified form of the verb くれる [to give]), to form the ending "-てください". 

If you want to sound rude, you can add the ending "くれ", to form the ending "-てくれ". 

There is one exception to this rule, which is the verb くれる. If you want to make a request with this verb, you instead use "くれ" (rude), "ちょうだい" (informal), and "ください" (formal), for example, when ordering food. 

The -て form can also be used to link verbs to each other. If you want to have two verbs in a sentence, like "I study and I play", put the first verb in -て form. You can also chain multiple verbs in -て form to create an ordered succession. 

\subsubsection{Conjugation}

For 一段 verbs, just add -て to the end of the -ます stem to get the -て form. 

For 五段 verbs, chop off the last syllable of the verb, and add endings depending on what you cut off, and below is a table:

\begin{tabular}{| c | c |}
\hline
Verb Ending: & Add:\\
\hline\hline
う,る,つ & -って\\
む,ぶ,ぬ & -んで\\
く & -いて\\
Exception: 行く & 行って\\
ぐ & -いで\\
す & -して\\
Exception: する & して\\
Exception: くる & きて\\
\hline
\end{tabular}

\begin{example}
    \ 

あれ見て

右に曲がってください

魚ください

私は勉強して、寝ます

普通に会社に就職して、普通の男性と恋をして、普通に結婚して家庭を築き、普通に子供を生んで幸せに暮らす
\end{example}

\begin{solution}
Look over there.

Please turn right.

I'll have the fish please.

I study and I sleep.

First, you get a job normally at a company, then you fall in love with an ordinary guy, then you get married like normal and raise a family, then you give birth to a child as normal and live a happy life.
\end{solution}

\subsection{-まして: Polite Te-form}

\subsubsection{Usage}

This is the "polite" Te-form, but in modern Japanese, it is not used very much. The regular "informal" Te-form is more than polite enough to be used in almost every situation.

\subsubsection{Conjugation}

For both 一段 and 五段 verbs, just take the -ます stem and add -まして to the end. 

For the two irregular verbs する and くる, they become しまして and きまして respectively.
\begin{example}
\ 

ヒューエットならわかってくれる気がしましてね
\end{example}

\begin{solution}
\ 

I had a feeling that Hewett would understand. 
\end{solution}

\subsection{-た: Informal Past Tense}
\subsubsection{Usage}

The informal past tense form is used to express that you did something in the past in an informal way.

\subsubsection{Conjugation}

To conjugate in the informal past tense, take the "-て" form of a verb and just replace the ending -て with a -た or the ending -で with a -だ.

\begin{example}
    \ 

図書館に行った

ここに財布があったか?

パンを食べた
\end{example}

\begin{solution}
    \ 

I went to the library.

Was there a wallet here?

I ate bread.
\end{solution}

\subsection{-ました: Polite Past Tense}

\subsubsection{Usage}

-ました is used to express what you did in a formal manner.

\subsubsection{Conjugation}

To conjugate, simply take the formal "-まして" form of any verb and replate ーて with -た.

\begin{example}
    \ 

今日、ごぜん十一時に起きました

図書館に勉強して、音楽を聞きました
\end{example}

\begin{solution}
    \ 

Today, I woke up at 11:00 AM

At the library, I studied and listened to music. 
\end{solution}

\subsection{-ている: Informal Continuous Form}

\subsubsection{Usage}

-ている is used to describe an action that is ongoing, like "I am eating", or a state of being, like "He is dead".

\subsubsection{Conjugation}

Just take the -て form of any verb, and add "いる" to the end. The verb "いる" is like "ある", in that it is used to describe something that exists, but it is used for animate things. 

If you want to further conjugate the verb, for example, to say you "aren't doing something", or to put it in past tense, you would conjugate "いる". For the first example, you would say, "-ていない", putting it in negatiev form, and for the second example, you would say "-ていだ", putting it in past tense.

In more casual settings, the "て" and the "い" can be merged to form "-てる".
\begin{example}
\ 

私わ食べている

お前わもう死んでいる

何を見ていだか?

\end{example}

\begin{solution}
\ 

I am eating

You are already dead

What was I seeing?
\end{solution}

\subsection{-ています: Polite Continuous Form}

\subsubsection{Usage}

The polite progressive/continous form is just a more formal way of using the continuous form.

\subsubsection{Conjugation}

Take a verb in the informal progressive form, and conjugate the ending verb "いる" into "います".

\begin{example}
\ 

私わ夢を見ていますか?

急いでいます

今、話しています

\end{example}

\begin{solution}
\ 

Am I seeing a dream?

I'm in a hurry

I'm talking right now. 
\end{solution}

\subsection{-ない: Informal Negative Form}

\subsubsection{Usage}

-ない mirrors the dictionary form in usage, but instead negates the verb, so it can be used to express that I am not doing something, that I do not habitually do something, or that I will not do something in the future.

\subsubsection{Conjugation}

For 一段 verbs, simply drop the ending "る" and add a "-ない" in its place. For 五段 verbs, change the ending vowel to an "あ" and add a "ない" on the end. In this case, a standalone "う" will become a "わ". 

There are three exceptions to this rule. The two irregular verbs する and くる become しない and こない respectively, and the verb ある becomes just ない.

\begin{example}
\ 

汚いから、中身みなでよ

どうするの、食べるの、食べないの?

もちろん話せないわ
\end{example}

\begin{solution}
\ 

It's dirty, don't look inside please (merged with て form)

What will you do? Eat, or not eat? (の is an informal alternative to が)

Of course I won't tell anyone (わ is added to make the sentence more feminine)
\end{solution}

\subsection{-ません: Polite Negative Form}

\subsubsection{Usage}

The polite negative form is the the negation of "-ます" and is used to express that you're not doing something now, you don't habitually do something, or that you won't do something in the future.

By adding a か to the end, and phrasing it as a question, you can politely suggest to do something.

\subsubsection{Conjugation}

Put the verb in "-ます" form and just replace it with "-ません". There is no exception for ある in this form.

\begin{example}
    \ 

アイスクリムを食べませn

明日勉強しません

郵便局に行きませんか?
\end{example}

\begin{solution}
\ 

I did not eat the ice cream.

I will not study tomorrow.

Should we go to the post office?
\end{solution}

\subsection{-なかった: Informal Past Negative Form}

\subsubsection{Usage}

-なかった is used to express something that didn't happen in the past.

\subsubsection{Conjugation}

Since the ending "ない" is an adjective, we can conjugate it into the past tense by removing the "い" and adding "-かった" to the end. Thus, the ending becomes "-なかった". Remember to swap the last vowel of 五段 verbs to "あ" too.

\begin{example}
\ 

昨晩、寝なかって.

いいえ、バス停に行かなかった.

先週、遺愛物しなかった.
\end{example}

\begin{solution}
\ 

I didn't sleep last night

No, I didn't go to the bus stop

Last week, I didn't go shopping
\end{solution}

\subsection{-ませんでした: Polite Past Negative Form}

\subsubsection{Usage}
This is used to express something that happened in the past.

\subsubsection{Conjugation}

でした is the past tense form of です, so by putting it at the end of a polite negative verb, we express that that verb happened in the past.

\begin{example}
\ 

いいえ、見ませんでした
\end{example}

\begin{solution}
\ 

No, I haven't seen it.
\end{solution}

\subsection{-れる: Informal and Polite Receptive Form}

\subsubsection{Usage}

The receptive form is for when you want to specify the recieving end of a verb but not nessecarily state who caused it. For example, you might use this to say "Sakura got scolded", or "I do not like it when I am looked at". In both cases, we don't really specify who is doing the verb (though we can), and instead, we just specify the reciever.

You can also use this form to describe a reciever of negative action, like I got my bag stolen.

\subsubsection{Conjugation}

For both 一段 and 五段 verbs, leave the verb in its dictionary form, and change the last vowel to an あ (standalone う goes to わ). DO NOT DROP THE る! Then, on the end, add the receptive helper verb れる. 

If you want to chain helpers, modify the れる, it works like an 一段 verb. For example, to put it in past tense, you would use "-れた". To make it polite, just use "-れます", and for polite past negative, you can use "-れませんした"

The reciever is the subject of the verb, so it will get a は or が paticle. If you want to specify the doer (optional), just give it a に particle.

The irregular form くる goes to こられる and the irregular form する goes to される
\begin{example}
\ 

さくらさんが叱られる.

蜂に刺されませんした.

水が飲まれた

さくらが鞄を誰かに盗まれた
\end{example}

\begin{solution}
\ 

Sakura is getting scolded.

I was stung by a bee.

The water got drunk (not intoxicated).

Someone stole Sakura's bag.
\end{solution}

It is important to note that there are two distinct verbs now, the original verb, and "れる". When you attach "れる", it becomes the main verb, and as such, the subject belongs to it, not the original verb. If we do specify a doer, that will be the subject for the original verb. 

Since the two subjects are different, we can't combine the verbs! (We can shorthand combine them for some cases like in the potential form) 

\subsection{-させる: Informal and Polite Causative Form}

\subsubsection{Usage}

This form is used to convey that someone let something happen or someone made something happen. 

\subsection{Conjugation}

For 一段 verbs, replace the ending る with させる. For 五段 verbs, replace the ending う with あ (remember う goes to わ), and append a せる.

くる turns into こさせる, and する goes to just させる.
\begin{example}
\ 

先生は私がトイレに行かせてください

昨日、お母さんはPS5をあそばせました

お父さんは映画を見させています
\end{example}

\begin{solution}
\ 

Teacher, please let me go to the toilet

Yesterday, Mom let me play on the PS5

Dad is letting me watch this movie

\end{solution}

\subsection{-させられる: Causative-Receptive Form}

\subsubsection{Usage}

This form expresses that you were made/coerced to do something.

\subsubsection{Conjugation}

This form follows the conjugation of the causative form.

\begin{example}
\ 

図書館に行かせられました

皿をあらわされて、私の部屋にそじ掃除させられました
\end{example}

\begin{solution}
\ 

I was made to go to the library

I was made to wash the dishes, then I was made to clean the room.
\end{solution}

In the second sentence, although あるわさせられて is technically correct, it is common to abbreviate せられる into される.

\subsection{-ろ/-え: Informal/Strong Imperative Form}

\subsubsection{Usage}

This form is used for a strong imperative, if something is urgent, or you want to strongly command someone to do something

\subsubsection{Conjugation}

For 一段 verbs, replace the ending る with ろ. For 五段 verbs, replace the last vowel う with え.

The irregular forms くる and する become こい and しろ respectively.
\begin{example}
\ 

待て!

やめろ!

行け!
\end{example}

\begin{solution}
\ 

Stop in place! (まって is the softer て-form request)

Stop doing what you are doing!

Go!
\end{solution}

\subsection{-なさい: Weak Imperative}

\subsubsection{Usage}

This is a comparatively weaker form of the strong imperative, but it is still not recommended for polite usage. Use the -てください ending for polite imperatives. 

The weak imperative is used for parent-child relations, for commands on tests, or for a caregiver-like tone in close relationships, like if your boyfriend is eating something that fell on the floor, you can say "食べなさい".

\subsection{Conjugation}

For 一段 verbs, drop the る and add なさい. 

For 五段 verbs, change the last vowel to い and add なさい.

\begin{example}
\ 

やめてなさい
\end{example}

\begin{solution}
\ 

Stop it!
\end{solution}

\subsection{-ば: Conditional Form}

\subsubsection{Usage}

This form is used to express if/when you do something. For example, in "When you see the convenience store, turn right", the first verb "When you see" would get the conditional れば form.

\subsubsection{Conjugation}

For 一段 verbs, replace the ending る with れば.

For 五段 verbs, replace the last う with an え, and then append え, and then append ば.

For the two irregular forms くる and する, they become くれば and すれば respectively.
\begin{example}
\ 

コンビニが見れば、左にまがってください

財布を渡せば、あなたに傷つけない

どうすれば小さくなれるなんだ?
\end{example}

\begin{solution}
\ 

When you see the convenience store, turn to the left.

If you give me the wallet, I won't hurt you.

What can I do to become small?
\end{solution}

Note: For the last example, asking what conditions I need to fulfill for another verb to take effect looks like "If I did what, can I become small"?

\subsection{-たら: Conditional Form}

\subsubsection{Usage}

This form is a conditional if/when an action occurs, but emphasizes what comes after the condition is met, for example "I will go home when school ends".

This form can also be used to describe counterfactuals, like "If I had money, I would eat McDonalds every day"

This form can not be used to describe disparate situations you have control over, like "I played after studying", because it implies that the result comes naturally after the condition, and in the above case, it doesn't. However, you can say "I stretch after running", because there is a natural causality, running and stretching are two related topics.

\subsubsection{Conjugation}

Take the -た past tense form of a verb and add ら to it.

For the two irregular forms くる and する, they become きたら and したら respectively.
\begin{example}
\ 

雨が降ったら、家にいます。

彼が来たら、話しましょう。

仕事が終わったら、映画を見に行こう。

転生したらスライムだった件
\end{example}

\begin{solution}
\ 

If it rains, I will stay home.

If he comes, let's talk.

When the work is finished, let's go watch a movie.

That Time I was Reincarnated as a Slime
\end{solution}

Note: For the last example, it actually reads "When I was reincarnated, I was a slime", and the 件 (けん) at the end is a marker used for an incident or a noteworthy event.

\subsection{-れる: Informal Potential Form}

\subsubsection{Usage}

The informal potential form expresses that you can do something, that you have the skill to do something, or that an object has -bility (edibility)



\subsubsection{Conjugation}

For 一段 verbs, remove the ending る and add られる. Since this overlaps with the receptive form, in informal speech, this is sometimes just contracted to "-れる". To turn this into the polite form, conjugate "られる" into "られます", and to negate it, it turns into "られない", or a polite negation would be "られません".

For 五段 verbs, change the final う sound to its corresponding え sound, then add る. Again, to turn this into different forms, just appropriately conjugate the ending "る".

The two irregular verbs くる and する become こられる and できる respectively.



\begin{example}
\ 

見られない

その魚は食べられるか?

日本語が話せますか?

聞けて、見られますか?
\end{example}

\begin{solution}
\ 

I can't see!

Is that fish edible?

Can you speak Japanese?

Can you hear and see me?
\end{solution}

Note: This form can not be used to express that things can happen without volition. For example, this form cannot be used to say "We could still lose". Instead, you would say "There are times where we lose", or "まだ負けることができる". In the above sentence, "まだ" means "still" or "yet", "負ける" means to lose, and "こと" turns that into a noun, and "できる" is the potential form of する. 

Phrased as a question, this form purely asks if someone is able to do something, but if you want to phrase a request, you can use the て-form and add the verb くれる, meaning to give (from the recipients perspective [part of a set of six giving verbs]). An example would be "五時にきてくれる?", or "Can you come over at 5?"

\subsection{-よう: Volitional Form}

\subsubsection{Usage}

The volitional form is expressing your willingness to do something, and its often translated as "Let's X", though not always.

\subsubsection{Conjugation}

For 一段 verbs in informal speech, replace the ending る with よう.

For 五段 verbs in informal speech, replace the last う with お, and append an う.

For polite speech, take the ます form, and replace the ending す with しょう, making -ましょう

\begin{example}
\ 

飲もう!
\end{example}

\begin{solution}
\ 

Let's Drink!    
\end{solution}

Note: In Bocchi the Rock, Kita Ikuyo's name is a pun of きた, meaning I have come, or I've arrived, and 行くよ, meaning let's go. Her last name isn't conjugated in the volitional form though, its the dictionary form appended with the particle よ, emphasizing it, so it has the same meaning. 行くよ is more commonly used in quite informal speech whereas the volitional conjugation 行こよ is used in only somewhat informal speech, and the polite form 行きましょう is used in polite speech. 

\section{Adjectives}

\subsection{-い Adjectives}

-い adjectives are one of two types of adjectives in Japanese. -い adjectives actually have the verb "to be" bundled inside of them, so they can be used standalone.

For example, "美味しい" is a perfectly valid sentence meaning "It is delicious". As such, these adjectives can end sentences, but can not be used with the だ copula as the meaning is duplicated

One drawback is that you need to conjugate these adjectives with politeness and tense.

\subsubsection{Present Informal}

Just simply leave the adjective as is.

To negate, simply turn the ending -い into -くない
\begin{example}
\ 

この魚は美味しい

支払いは悪くない
\end{example}

\begin{solution}
\ 

This fish is delicious!

The payment isn't bad.
\end{solution}

\subsubsection{Present Formal}

We can actually add -です to the end of -い adjectives to make it more formal, although we aren't using the actual meaning of the copula, we are borrowing its politeness.

To negate, simply negate -です to make -ではありません
\begin{example}
\ 

そらはあおいです

料理することは悪いではありません
\end{example}

\begin{solution}
\ 

The sky is blue

The cooking isn't bad
\end{solution}
\end{CJK}
\end{document}