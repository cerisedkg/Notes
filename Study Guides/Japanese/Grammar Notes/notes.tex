\documentclass{article}
\usepackage{alexconfig}
\title{Japanese Grammar Notes}
\usepackage{CJKutf8}


\begin{document}
\maketitle

\begin{CJK}{UTF8}{min}

\section{Motivation}
This is an explainer document on all of the properties of Japanese grammar I have encountered.
\section{Sentence Structure}
Basic Japanese sentence structure goes subject, nouns/objects, and a final verb, is generally more freeform than English sentence structure, and that is because Japanese uses markers called particles attached to certain clauses.

\begin{definition}[Particles]
Particles, or 助詞(じょし), are markers placed after nouns, adjectives, and verbs, that modify their meaning, or clarify their usage.
\end{definition}

\begin{example}
メアリさん\textbf{は}図書館\textbf{で}二時\textbf{に}勉強します
\end{example}
As we can see above, the highlighted kana (which means hiragana \& katakana) above are a few examples of the many many particles in Japanese, and we will go into more detail in the particles section. 

Japanese sentences have a train-car structure, where clauses and even verb conjugations can "link" one after another, and the language overall is very modular. 

Not every clause needs to have a particle, however. Propositional phrases like "みぎ" (in front of), or like "二時間" (two hours in duration), can stand independently, and these are called nude nouns.

In addition, while English is a stickler for fully formed thoughts in sentences, Japanese is much more lax. For example, take this sample conversation:

\begin{example}
たけし:メアーリさんはカフェでコーヒーをのみましたか?
ロバト:いいえ、お茶をのみました。
\end{example}

In the example above, we see that since Takeshi-san is starting the conversation, he provides much of the needed context, like where and what he is inquiring about, but Robert-san replies with a very stripped-down reply. Directly translated, we have:

\begin{example}
Takeshi-san: Did Mary-san drink coffee at the cafe?
Robert-san: No, drank tea.
\end{example}

As opposed to saying "No, Mary drank tea at the cafe", which would be an appropriate response in English, instead we strip all of the fluff out. Japanese conversations resemble lego blocks, each person building off ideas from the last, not presenting already wholly formed ideas like in English. 

\section{Particles}
We detail some common particles and their usages below:

\subsection{は}
は when used as a particle is pronounced "wa", and is the topic marker. It sets the topic for a sentence or conversation, but is not always the subject of the sentence. 

\begin{example}
メーアリさんの選考\textbf{は}経済です。
\end{example}

In this sentence, we are talking about Mary-san's major, so we mark that with the は particle.

\begin{example}
    テレビ\textbf{は}子供が見ます.
\end{example}
This sentence translated is "The child watches the television", but as we can see, the topic is not the subject. The subject is the child, but we are talking about the television, so it is marked as the topic. 

%Wa shifts focus to the verb, ga shifts focus to the subject, thats why we use ga with arimasu, we care more about the subject (the thing which exists), rather than the action of existing itself.
\section{Verbs and Conjugation}

\end{CJK}
\end{document}