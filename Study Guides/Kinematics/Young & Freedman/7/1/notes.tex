\documentclass{article}
\usepackage{alexconfig}
\title{7.1: Gravitational Potential Energy}

\begin{document}
\maketitle
\begin{definition}[Potential Energy]
Energy associated with position is called \textbf{potential energy}.
\end{definition} 

\begin{proposition}
$W_{grav} = \vec{F}\cdot \vec{s} = \vec{F_g}(y_1 - y_2 = mgy_1 - mgy_2)$

The above derivation shows that when an object moves upward, negative work is done, and potential energy increases, and when a body moves downwards, positive work is done, and potential energy decreases. 

$$U_{grav} = mgy$$

$$W = -\Delta U_{grav}$$
\end{proposition}

\begin{theorem}[Conservation of Mechanical Energy]
If the only force doing work on a system is gravity, then mechanical energy is conserved, in other words: $$K_0 + U_0 = K_1 + U_1$$
\end{theorem}

If forces other than gravity do work, then our equation becomes $$K_0 + U_0 + W = K_1 + U_1$$

\begin{proposition}[Gravitational Potential Energy along a Curved Path]
Remember that $-\Delta U_{grav} = W$, so $m\vec{g} \cdot \vec{s} = -\Delta U_{grav}$. This means we only consider the component of displacement that is inline with the force of weight. 
\end{proposition}
\end{document}