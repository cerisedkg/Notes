\documentclass{article}
\usepackage{alexconfig}
\title{8.5: Center of Mass}

\begin{document}
\maketitle
\begin{definition}[Center of Mass]
The \textbf{center of mass} of a system of particles is the average position of the particles weighted by mass:

$$\vec{r}_{cm} = \frac{m_1 \vec{r}_1 + m_2 \vec{r}_2 + m_3\vec{r}_3 + ...}{m_1 + m_2 + m_3 + ,,,}$$
\end{definition}

Taking the time derivative with respect position, we get $$\vec{v}_{cm} = \frac{m_1 \vec{v}_1 + m_2\vec{v}_2 + m_3\vec{v_3} + ...}{m_1 + m_2 + m_3 + ...}$$

Multiplying the denominator to the other side, we get $$(m_1 + m_2 + m_3 + ...)\vec{v}_{cm} = m_1\vec{v}_1 + m_2\vec{v_2} + m_3\vec{v_3} + ...$$And rewriting, we get $$\vec{p}_{cm} = \vec{p}_1 + \vec{p}_2 + \vec{p}_3 + ...$$Or the momentum of the center of mass is equal to the sum of the momentum of each particle in the system.

If we take the time derivative of our velocity equation again, we see that $$(m_1 + m_2 + m_3 + ...)\vec{a}_{cm} = m_1\vec{a}_1 + m_2\vec{a_2} + m_3\vec{a_3} + ...$$Or the net force acting on the center of mass is equal to the sum total of all forces acting on all particles.

However, the internal forces of the system cancel each other, so actually, $$\vec{F}_{ext} = (m_1 + m_2 + m_3 + ...)\vec{a}_{cm}$$Or the force acting on the center of mass is the sum of all external forces acting on the system.
\end{document}