\documentclass{article}
\usepackage{tikz}
\usepackage{parskip}
\usepackage{xcolor}
\usepackage{textcomp, gensymb}
\usepackage{pgfplots}
\usepackage{tkz-euclide}
\usepackage[bottom=0.5in,top=0.5in,right=0.5in,left=0.5in]{geometry}
\usepackage{amsmath}
\usepackage{amsfonts}
\usepackage{amssymb}
\usepackage{enumitem}
\usepackage{amsthm}
\pgfplotsset{compat=1.18}
\title{3.3: Polar Representation of Complex Numbers}
\author{Alex L.}
\date{\today}
\pagecolor[rgb]{0,0,0} %black
\color[rgb]{1,1,1} %white

\begin{document}
\maketitle

Previously, we were using the cartesian system of complex coordinates, but now its time for another form, which has different use cases.

\subsection{Definition}

\textbf{Def:} The \textbf{polar form} of a complex number is the form $$z = re^{i\theta}$$where $r = \vert z\vert$ and $\theta = \arg z$. This form makes the modulus and argument of a complex number clear.

\subsection{Multiplication and Division}

$z_1 z_2 = r_1e^{i\theta_i}r_2e^{i\theta_2} = r_1r_2 e^{i(\theta_1 + \theta_2)}$

$\frac{z_1}{z_2} = \frac{r_1}{r_2}e^{i(\theta_1 - \theta_2)}$

\end{document}