\documentclass{article}
\usepackage{tikz}
\usepackage{parskip}
\usepackage{xcolor}
\usepackage{textcomp, gensymb}
\usepackage{pgfplots}
\usepackage{tkz-euclide}
\usepackage[bottom=0.5in,top=0.5in,right=0.5in,left=0.5in]{geometry}
\usepackage{amsmath}
\usepackage{amsfonts}
\usepackage{amssymb}
\usepackage{enumitem}
\usepackage{amsthm}
\pgfplotsset{compat=1.18}
\title{4.1: Series}
\author{Alex L.}
\date{\today}
\pagecolor[rgb]{0,0,0} %black
\color[rgb]{1,1,1} %white

\begin{document}
\maketitle

\textbf{Def:} A \textbf{series} is an addition of infinitely many terms, often related to each other. A \textbf{partial sum} is the summation of the first $n$ numbers in a series.

\textbf{Def:} A series is said to \textbf{converge} if the partial sum of the first $n$ terms as $n$ approaches infinity approaches a finite number. If not, it is said to \textbf{diverge}. If the series has a variable, convergence may depend on the value that variable takes on.

\textbf{Def:} A series may contain complex numbers, in which case the real and complex components of each term can be broken apart and added separately, in the form $A+iB$, where $A$ and $B$ are the sub-series of the real and complex terms respectively. A complex series converges if both $A$ and $B$ converge. 

\end{document}