\documentclass{article}
\usepackage{tikz}
\usepackage{parskip}
\usepackage{xcolor}
\usepackage{textcomp, gensymb}
\usepackage{pgfplots}
\usepackage{tkz-euclide}
\usepackage[bottom=0.5in,top=0.5in,right=0.5in,left=0.5in]{geometry}
\usepackage{amsmath}
\usepackage{amsfonts}
\usepackage{amssymb}
\usepackage{enumitem}
\usepackage{amsthm}
\pgfplotsset{compat=1.18}
\title{4.3: Convergence of a Series}
\author{Alex L.}
\date{\today}
\pagecolor[rgb]{0,0,0} %black
\color[rgb]{1,1,1} %white

\begin{document}
\maketitle

If the partial sum of the first $n$ terms of a series tends towards a finite number as $n$ approaches infinity, the series is said to converge.

\subsection{Absolute and Conditional Convergence}

\textbf{Def:} Suppose we have an infinite series $\sum u_n$, where $u_n$ are the terms of the series which can be complex, real, positive, or negative. If the series $\sum \vert u_n \vert$ converges (the modulus of all terms), then the series $\sum u_n$ will converge also. This type of series is called \textbf{absolutely convergent}. For an absolutely convergent series, you may rearrang the terms without affecting its convergence. 

\textbf{Def:} In some cases, $\sum \vert u_n \vert$ will diverge while $\sum u_n$ will converge. These series are called \textbf{conditionally convergent}. Rearranging the terms of a conditionally convergent series will affect its convergence.

\subsection{Tests for Series with Real Positive Terms}

A nessecary, but not sufficient, criteria is that the $n$th term of a positive real series must tend towards zero as $n$ tends towards infinity. That is to say, $$\lim_{n \to \infty} u_n = 0$$

If this condition is not satisfied, the series will diverge.

\textbf{Def:} The \textbf{comparison test} is the most basic test for convergence. Suppose we have a series $\sum v_n$ that we know converges, and a series $\sum u_n$ which we want to test. If every term of $u_n \leq v_n$ for some $n>N$, then we know that $\sum u_n$ converges. The exact $N$ where this comparison starts mattering differs from series to series. 

\textbf{Def:} The \textbf{ratio test} determines convergence by comparing ratios. If $$\lim_{n \to \infty}(\frac{u_{n+1}}{u_n}) < 1$$then the series is convergent. If it is greater than one, the series is divergent. If it is equal to 1, the test is inconclusive.  

\textbf{Def:} The \textbf{ratio comparison test} combines the two former tests. Let us consider two series $\sum u_n$ and $\sum v_n$, and we know that $\sum v_n$ is convergent. If the ratio $\frac{u_{n+1}}{u_n} \leq \frac{v_{n+1}}{v_n}$, then $\sum u_n$ is convergent. The same can be said if the ratio is greater than that of a known divergent series. 

\textbf{Def:} Another way to combine the ratio and comparison test is the \textbf{quotient test}. Let us have $\sum u_n$ and $\sum v_n$, and take the limit $$\lim_{n \to \infty}(\frac{u_n}{v_n})$$If this limit is nonzero and finite, then both series converge or diverge. If the limit is equal to zero, then they both converge, and if its infinity, they both diverge. 

\textbf{Def:} The \textbf{integral test} is another test. We want to test the convergence of $\sum u_n$. Suppose there is some function $f(x)$ that constantly decreases for $x$ greater than some value $x_0$, and for which, $f(n) = u_n$, that is, the value of the function at $n$ corresponds to the value of the series. Then, if the limit of the integral $$\lim_{N \to \infty} \int^{N} f(x) dx$$ exists, then the series is convergent, otherwise, it diverges. 

\textbf{Def:} The \textbf{root test} starts by defining a limit $$\lim_{n\to\infty}(u_n)^{\frac{1}{n}}$$then the series $\sum u_n$ converges if the limit is less than $1$, and if the limit is greater than $1$, the series diverges. The test is inconclusive if the limit equals $1$. 

We can also group terms to turn series into more comparable forms, and then perform the above tests. 

\subsection{Tests for Alternating Series}

An alternating series will converge if $u_n$ tends towards zero as $n$ approaches infinity, and each successive term is smaller in magnitude than the previous term, starting at some finite $N$. If these conditions aren't met, then the series will oscillate. 
\end{document}