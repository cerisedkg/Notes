\documentclass{article}
\usepackage{tikz}
\usepackage{parskip}
\usepackage{xcolor}
\usepackage{textcomp, gensymb}
\usepackage{pgfplots}
\usepackage{tkz-euclide}
\usepackage[bottom=0.5in,top=0.5in,right=0.5in,left=0.5in]{geometry}
\usepackage{amsmath}
\usepackage{amsfonts}
\usepackage{amssymb}
\usepackage{enumitem}
\usepackage{amsthm}
\pgfplotsset{compat=1.18}
\title{4.4: Operations with Series}
\author{Alex L.}
\date{\today}
\pagecolor[rgb]{0,0,0} %black
\color[rgb]{1,1,1} %white

\begin{document}
\maketitle

$\sum ku_n = k(\sum u_n)$

If $\sum u_n = S$ and $\sum v_n = T$ then $\sum(u_n+v_n) = S+T$

If $\sum u_n = S$ then $a+\sum u_n = a+S$. This indicates that the insertion of a finite number of terms anywhere in the series doesn't affect its convergence.

If infinite series $\sum u_n$ and $\sum v_n$ are both convergent, then the series $w_n = u_1v_n + u_2 v_{n-1} + u_3 v_{n-2} + ... + u_n v_1$ is also convergent. This operation is called the \textbf{Cauchy product} of the two series.

In general, term by term integration or differentiation will not always preserve convergence or divergence. 

\end{document}