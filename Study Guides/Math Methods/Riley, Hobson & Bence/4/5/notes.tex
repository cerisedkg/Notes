\documentclass{article}
\usepackage{tikz}
\usepackage{parskip}
\usepackage{xcolor}
\usepackage{textcomp, gensymb}
\usepackage{pgfplots}
\usepackage{tkz-euclide}
\usepackage[bottom=0.5in,top=0.5in,right=0.5in,left=0.5in]{geometry}
\usepackage{amsmath}
\usepackage{amsfonts}
\usepackage{amssymb}
\usepackage{enumitem}
\usepackage{amsthm}
\pgfplotsset{compat=1.18}
\title{4.5: Power Series}
\author{Alex L.}
\date{\today}
\pagecolor[rgb]{0,0,0} %black
\color[rgb]{1,1,1} %white

\begin{document}
\maketitle
\textbf{Def:} \textbf{Power series} are series in the form $$P(x) = a_0 + a_1x + a_2x^2 + a_3 x^3 + ...$$

The ratio test shows us that if $$\lim_{n \to \infty} \vert \frac{a_{n+1}}{a_n}x \vert <1 \rightarrow \vert x\vert \lim_{n \to\infty}\vert \frac{a_{n+1}}{a_n} \vert<1$$the series converges. 

As you can see, the convergence depends on $x$, so we will often get an \textbf{interval of convergence} where $P(x)$ converges.

\subsection{Operations}

If $P(x)$ and $Q(x)$ are power series, then the interval of convergence of $P(x) \pm Q(x)$ is the overlap of the two intervals of convergence.

If two power series converge for all values $x$, you may substitute one into the other to get another series that converges for all values $x$. For example, $e^x$ and $\sin x$ both converge for all $x$, and so does $e^{\sin x}$.

The derivative or integral of a power series also converges in the same range.


\end{document}