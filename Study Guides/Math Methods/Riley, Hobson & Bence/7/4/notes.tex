\documentclass{article}
\usepackage{tikz}
\usepackage{parskip}
\usepackage{xcolor}
\usepackage{textcomp, gensymb}
\usepackage{pgfplots}
\usepackage{tkz-euclide}
\usepackage[bottom=0.5in,top=0.5in,right=0.5in,left=0.5in]{geometry}
\usepackage{amsmath}
\usepackage{amsfonts}
\usepackage{amssymb}
\usepackage{enumitem}
\usepackage{amsthm}
\pgfplotsset{compat=1.18}
\title{7.4: Basis Vectors and Components}
\author{Alex L.}
\date{\today}
\pagecolor[rgb]{0,0,0} %black
\color[rgb]{1,1,1} %white

\begin{document}
\maketitle

In 3D space, given three non-coplanar vectors $e_1, e_2, e_3$, it is possible to describe any vector in 3D space in the form $a = a_1e_1 + a_2e_2 + a_3e_3$.

\textbf{Def:} The \textbf{basis} of an $n$-dimensional space is a set of $n$ linearly independent vectors such that every vector in the space may be described as a linear combination of the vectors. The coefficients of the basis vectors are called \textbf{components}. 

\textbf{Def:} A \textbf{linear combination} of quantities $x_1, x_2, x_3, ... ,x_n$ is a quantity $c = a_1x_1 + a_2x_2 + a_3x_3 + ... + a_nx_n$, where $a_1, a_2, a_3, ... , a_n$ are scalar quantities.  

\textbf{Theorem:} Any set of $n$ linearly independent vectors forms a basis for an $n$-dimensional space. 

By convention, in 3D space, we use the vectors, $\vec{i}, \vec{j}, \vec{k}$, which align with the $x,y,$ and $z$ axes respectively. However, for brevity, the coefficients of the linear combination are usually expressed in terms of its components only: $(a_i, a_j, a_k)$. 

The sum of two vectors is the sum of its components.



\end{document}