\documentclass{article}
\usepackage{tikz}
\usepackage{parskip}
\usepackage{xcolor}
\usepackage{textcomp, gensymb}
\usepackage{pgfplots}
\usepackage{tkz-euclide}
\usepackage[bottom=0.5in,top=0.5in,right=0.5in,left=0.5in]{geometry}
\usepackage{amsmath}
\usepackage{amsfonts}
\usepackage{amssymb}
\usepackage{enumitem}
\usepackage{amsthm}
\pgfplotsset{compat=1.18}
\title{7.5: Magnitude of a Vector}
\author{Alex L.}
\date{\today}
\pagecolor[rgb]{0,0,0} %black
\color[rgb]{1,1,1} %white

\begin{document}
\maketitle
\textbf{Def:} The \textbf{magnitude} of an $n$-dimensional vector is denoted $\vert a\vert = \sqrt{a_1^2 + a_2^2 + a_3^2 + a_4^2 + ... + a_n^2}$, and represents the length of the vector. 

\textbf{Def:} A vector whose magnitude equals unity (1) is called a \textbf{unit vector}. A unit vector is denoted with a hat: $\hat{a}$, and is also called a normalized vector. To normalize a vector, divide it by its magnitude: $\frac{\vec{a}}{\vert a\vert}$. 


\end{document}