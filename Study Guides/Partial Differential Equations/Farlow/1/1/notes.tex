\documentclass{article}
\usepackage{alexconfig}
\title{1.1: Introduction to Partial Differential Equations}

\begin{document}
\maketitle
\ 
\begin{definition}[Partial Differential Equations]
\textbf{Partial differential equations}, also known as PDEs, are differential equations with partial derivatives. The solutions to these equations will contain more than one independent variable.

We denote $\frac{\partial u}{\partial t}$ as $u_t$, and $\frac{\partial^2 u}{\partial t^2}$ as $u_{tt}$ and such for other variables    
\end{definition}

\begin{example}[Common PDEs]
    \ 
\begin{enumerate}
    \item Heat Equation (1D): $u_t = u_{xx}$
    \item Heat Equation (2D): $u_t = u_{xx} + u_{yy}$
    \item General Heat Equation: $u_t = \nabla^2 u$
    \item General Wave Equation: $u_{tt} = \nabla^2 u$
    \item Telegraph Equation: $u_{tt} = u_{xx} + \alpha u_t + \beta u$
    \item Laplace's Equation in Polar Coordinates: $u_rr + \frac{1}{r} u_r + \frac{1}{r^2} u_{\theta\theta} = 0$
\end{enumerate}
\end{example}

There are generally ten methods we use to solve PDEs:

\begin{enumerate}
    \item Separation of Variables: We reduce a PDE in $n$ variables to a system of $n$ ODEs
    \item Integral Transforms: We can reduce a PDE in $n$ variables to a PDE in $n-1$ variables
    \item Change of Coordinates: Can simplify a PDE or potentially turn it into an ODE
    \item Transformation of the Dependent Variable: We substitute the dependent variable with one that is easier to find.
    \item Numerical Methods: Changes a PDE into a system of \textbf{difference equations} which can be brute forced by a computer
    \item Perturbation Methods: This methods changes a nonlinear PDE into a system of linear PDEs which \textit{approximate} the original PDE
    \item Impulse Response: Changes the boundary conditions (initial values) of a PDE into impulses (dirac delta function), and measures the responses to these impulses. These impulses can be reconstructed into the solution.
    \item Integral Equations: Changes a PDE to an integral equation (where the unknown is inside the integral), and then solves the integral with various techniques
    \item Calculus of Variations: Reframes a solution to a PDE as a minimization problem (think the action in Lagrangian mechanics or the total energy in Hamiltonian mechanics). The solution that minimizes the action/total energy is ususally the solution to the problem.
    \item Eigenfunction Expansion: We attemt to find the solution to a PDE as an infinte sum of eigenfunctions (functions that become scalar multiples of themselves when put in the PDE)
\end{enumerate}

\begin{definition}[Classifications of PDEs]
\ 
\begin{enumerate}
    \item Order: Order is the highest derivative that appears in the PDE
    \item Number of Variables: THe number of independent variables
    \item Linearity: If the dependent variable and its derivatives are linear (not multiplied by each other)
    \item Homogeneity: A PDE is homogeneous if there is no coefficient that is not multiplied by a derivative. (Only really matters for second order linear)
    \item Constant Coefficients: If all coefficients of derivatives are constant, the equation is a constant coefficients PDE
\end{enumerate}

In addition, second order linear PDEs in two variables of the form $$Au_{xx} + Bu_{xy} + Cu_{yy} + Du_x + Eu_y + Fu + G = 0$$have special properties:

\begin{enumerate}
    \item If $B^2 - 4AC > 0$, the equation is \textbf{hyperbolic} and represents vibrating systems and wave motion.
    \item If $B^2 -4AC = 0$, the equation is \textbf{parabolic} and describes heat and diffusion.
    \item If $B^2 - 4AC < 0$, the equation is \textbf{elliptic} and describes steady state systems.
\end{enumerate}
Notice: $A,B,C$ can be functions, and so these equations can switch types.
\end{definition}

\begin{exercise}
Classify the following PDES:
\begin{enumerate}
    \item $u_t = u_{xx} + 2u_x + u$
    \item $u_t = u_{xx} + e^{-t}$
    \item $u_{xx} + 3u_{xy} + u_{yy} = \sin(x)$
    \item $u_{tt} = uu_{xxxx} + e^{-t}$
\end{enumerate}
\end{exercise}

\begin{solution}
\ 
\begin{enumerate}
    \item Constant coefficient second order linear homogeneous parabolic PDE in two variables
    \item Constant coefficients second order linear parabolic PDE in two variables
    \item Constant coeffcients second order linear hyperbolic PDE in two variables
    \item Fourth order PDE in two variables
\end{enumerate}
\end{solution}

\begin{exercise}
How many solutions to the PDE $u_t = u_{xx}$ can you find? Try solutions of the form $u(x,t) = e^{ax + bt}$
\end{exercise}

\begin{solution}
If $$u(x,t) = e^{ax + bt}$$ then $$u_t = be^{ax+bt}$$and $$u_{xx} = a^2e^{ax+bt}$$Setting these equal to each other, we get that all solutions of the form $u(x,t) = e^{ax + bt}$ where $$b = a^2$$ are valid. 
\end{solution}

\begin{exercise}
If $u_1(x,y)$ and $u_2(x,y)$ are both solutions to $$Au_{xx} + Bu_{xy} + Cu_{yy} + Du_x + Eu_y + Fu = G$$then is $u_1 + u_2$ a solution?
\end{exercise}

\begin{solution}
When we say that $u_1$ and $u_2$ are solutions to the above equations, it means that when we plug them into the differential equation, we get back $G$. If we plug in $u_1 + u_2$, however, we get $$A(u_1+u_2)_{xx} + B(u_1 + u_2)_{xy} + C(u_1+u_2)_{yy} + D(u_1+u_2)_x + E(u_1+u_2)_y + F(u_1+u_2)$$Since $(u_1+u_2)_{xx} = (u_1)_{xx} + (u_2)_{xx}$ and so on for the other variables, we can separate the $u_1$ and $u_2$ to get $$A(u_1)_{xx} + B(u_1)_{xy} + C(u_1)_{yy} + D(u_1)_x + E(u_1)_y + F(u_1) + A(u_2)_{xx} + B(u_2)_{xy} + C(u_2)_{yy} + D(u_2)_x + E(u_2)_y + F(u_2)$$ We know that individually, each segment is equal to $G$, so the entire thing is equal to $2G$, therefore, $u_1 + u_2$ is not a solution.
\end{solution}

\begin{exercise}
Find a solution to $\frac{\partial u(x,y)}{\partial x} = 0$
\end{exercise}

\begin{solution}
The solution is $u(x,y) = f(y) +C$
\end{solution}

\begin{exercise}
Find a solution to $\frac{\partial^2 u(x,y)}{\partial x\partial y} = 0$  
\end{exercise}

\begin{solution}
The solution is of the form $u(x,y) = f(x) + f(y) + C$ 
\end{solution}

\end{document}