\documentclass{article}
\usepackage{tikz}
\usepackage{parskip}
\usepackage{xcolor}
\usepackage{textcomp, gensymb}
\usepackage{pgfplots}
\usepackage{tkz-euclide}
\usepackage[bottom=0.5in,top=0.5in,right=0.5in,left=0.5in]{geometry}
\usepackage{amsmath}
\usepackage{amsfonts}
\usepackage{amssymb}
\usepackage{enumitem}
\usepackage{amsthm}
\pgfplotsset{compat=1.18}
\title{3.2: Conditional Probabilities}
\author{Alex L.}
\date{\today}
\pagecolor[rgb]{0,0,0} %black
\color[rgb]{1,1,1} %white

\begin{document}
\maketitle
\subsection{Motivation}

Suppose we toss two dice. Each ordered pair of numbers on the dice has a $\frac{1}{36}$ chance of happening. Now, what if the first dice lands on $3$? Well, now we only have six outcomes, $(3,1),(3,2),...,(3,6)$, and each has a $\frac{1}{6}$ chance of occuring, and the probability of all other outcomes is now $0$. 

\subsection{Conditional Probabilities}

\textbf{Def:} If we let $E$ denote the event that the sum of rolling two fair dice is $8$, and $F$ denote the event that the first dice lands on a $3$, then the probability that $E$ will occur if $F$ just happened is denoted $$P(E \vert F)$$and is read "the probability of $E$ given $F$". These situations are called \textbf{conditional probabilities}.  

There is a formula for conditional probabilities.$$P(E\vert F) = \frac{P(E \cap F)}{P(F)}$$

\subsection{Examples}

\textbf{Example 2a:} A student is taking a one-hour-time-limit makeup examination. Suppose the probability that the student will finish the exam in less than $x$ hours is $\frac{x}{2}$, for all $0 \leq x\leq 1$. Then, given that the student is still working after $.75$ hour, what is the conditional probability that the full hour is used?

\textbf{Solution:} The event $E$ is when the full hour is used, and the event $F$ is when $.75$ the student hasn't finished in $.75$ hours. $P(E)$ is equal to $1-\frac{1}{2}$, as $\frac{1}{2}$ is the probability the student finishes in less than an hour, so $P(E) = \frac{1}{2}$. Likewise, $P(F)$ is the probability the student hasn't finished in $.75$ hours, which is equal to $1 - \frac{.75}{2} = .625$. However, $P(F)$ always happens when $P(E)$ happens, so $P(E \cap F) = P(E) = .5$. As such, we have $P(E\vert F) = \frac{.5}{.625} = .8$.

\textbf{Example 2b:} A fair coin is flipped twice. What is the probability that I flip two heads given I the first flip is heads? How about if at least one flip is heads?

\textbf{Solution:} Let $E = \{(H,H)\}$ be the event of flipping two heads, and let $F = \{(H,H),(H,T)\}$ be the event of the first coin flipping heads, and let $G = \{(H,H),(H,T),(T,H)\}$ be the event that at least one coin is heads. $P(E \vert F) = \frac{P(E \cap F)}{P(F)} = \frac{\{(H,H)\}}{\{(H,H),(H,T)\}} = \frac{1}{2}$

$P(E \vert G) = \frac{\{(H,H)\}}{\{(H,H),(H,T),(T,H)\}} = \frac{1}{3}$

\end{document}