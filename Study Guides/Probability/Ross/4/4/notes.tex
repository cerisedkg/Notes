\documentclass{article}
\usepackage{alexconfig}
\title{4.4: Expected Values of Functions of Random Variables}

\begin{document}
\maketitle
\section{Motivation \& Goals:}
Suppose I have a discrete random variable $X$, and a function $g(X) = Y$ transforming that random variable into another random variable $Y$? What is the expected value of $Y$?

\section{Content}
\ 
\begin{proposition}
If $X$ is a discrete random variable with values $x_i$, and with respective probabilities $p(x_i)$, then $E(g(X))$, the expected value of the transformed variable, can be calculated by $\sum_{i} g(x_i)p(x_i)$.
\end{proposition}
\begin{customproof}
The events in the original variable still occur in the same proportions, so $g(x_i)$ occurs just as frequently as $x_i$, as such, to get the expected value for $g_{X}$, we just take the expected value of $X = \sum_{i}x_ip(x_i)$ and swap $x_i$ for $g_(x_i)$ to get $E(g(X)) = \sum_{i}g(x_i)p(x_i)$
\end{customproof}

\begin{example}[4a]
Let $X$ denote a random variable with $P(\{X = -1\}) = .2 \ P(\{X=0\}) = .5 \ P(\{X = 1 \}) = 1$. What is $E(X^2)$?
\end{example}

\begin{solution}
$$E(X^2) = -1^2P(\{X=-1\}) + 0^2P(\{X=0\}) + 1^2P(\{X=1\}) = 1*.2 + .5+0 + 1*.3 = .5$$
\end{solution}

\end{document}