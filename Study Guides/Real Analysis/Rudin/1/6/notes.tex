\documentclass{article}
\usepackage{alexconfig}
\title{Rudin Chatper 1: The Complex Numbers}

\begin{document}
\maketitle

\begin{definition}[Complex Numbers]
\

\textbf{Complex numbers} are ordered pairs $(a,b)$ (ordered meaning $(a,b) \neq (b,a)$) where $a,b$ are real numbers. 

We say that complex numbers $x = (a,b)$ and $y = (c,d)$ are equal, $x=y$ if $a=c$ and $b=d$. 

We define addition of complex numbers and multiplication of complex numbers as follows:

\begin{enumerate}
    \item $x+y = (a+c, b+d)$
    \item $x\cdot y = (ac-bd, ad+bc)$
\end{enumerate}
\end{definition}

\begin{theorem}
\

The operations defined above turn the complex numbers into a field with $(0,0)$ as the additive identity and $(1,0)$ as the multiplicative identity.
\end{theorem}

\begin{customproof}
\

% closure under addition and multiplication, commutativity under addition and multiplication, associativity, distributive law, additive identity and inverses, and multiplicative identity and inverses.

Closure under addition is pretty obvious, if we have complex numbers $x = (a,b)$ and $y = (c,d)$ then $x+y = (a+c,b+d)$. Since $a+c$ and $b+d$ are both reals (due to closure under the reals), the sum will also be a complex number. 

Addition is commutative because $x+y = (a+c, b+d) = (c+a, d+b) = y+x$.

Addition is associative because $x+(y+z) = (a+b)+(c+e, d+f) = (a+c+e, b+d+f) = (a+c, b+d) + (e,f) = (x+y) + z$

$(0,0)$ is the additive identity since $x + (0,0) = (a+0, b+0) = (a,b) = x$ for all $x$ in the complex numbers.

For any complex number $x = (a,b)$, we let $-x$, the additive inverse of $x$, be $(-a,-b)$. Then, $x+ (-x) = (a + (-a), b + (-b)) = (0,0)$ and $(0,0)$ is the additive identity.

There is closure under multiplication because $x\cdot y = (ac-bd, ad+bc)$ and since both $ac-bd$ and $ad+bc$ are real numbers, the resulting product is a complex number.

Multiplication is commutative beacuse $x \cdot y = (ac-bd, ad+bc) = (ca-db, da+cb) = y \cdot x$

Multiplication is associative beacuse $x \cdot (y \cdot z) = (a,b) \cdot (ce-df, cf+de) = (ace-adf - bcf - bde, acf + ade + bce - bdf) = (ac-bd, ad+bc) \cdot (e,f) = (x \cdot y) \cdot z$

$(1,0)$ is the multiplicative identity since $x * (1,0) = (a,b) * (1,0) = (a*1-b*0, b*1 + a* 0 ) = (a,b)$

For any complex number $x=(a,b)$, we choose $x ^{-1}  = (\frac{a}{a^2+b^2}, -\frac{b}{a^2+b^2})$, and this is an inverse since $x \cdot x ^{-1} = (\frac{a^2}{a^2+b^2} + \frac{b^2}{a^2+b^2}, \frac{ab}{a^2+b^2}-\frac{ab}{a^2+b^2})  =(1,0)$

Distributive law holds because $x*(y+z) = (a,b) * (c+e, d+f) = (ac+ae-bd-bf, ad+af + bc+be) = x*y + x*z$
\end{customproof}

\begin{theorem}
\

For any real numbers $a,b$ we have $(a,0)+ (b,0) = (a+b,0)$ and $(a,0) \cdot (b,0) = (a*b,0)$
\end{theorem}

\begin{customproof}
\

$$(a,0) + (b,0) = (a+b, 0+0) = (a+b,0)$$

$$(a,0) \cdot (b,0) = (ab - 0*0, a*0+b*0) = (ab, 0)$$
\end{customproof}

\begin{definition}[i]
\

$$i = (0,1)$$
\end{definition}

Note: $(a,0) = a$


\begin{theorem}
\

$$i^2 = -1$$
\end{theorem}

\begin{customproof}
\

$$i^2 = i\cdot i = (0,1) \cdot (0,1) = (0*0 - 1*1, 0*1 + 0*1) = (-1,0) = -1$$
\end{customproof}

\begin{theorem}
\

If $a$ and $b$ are real, then $(a, b) = a+bi$.
\end{theorem}


\begin{customproof}
\

$$a+bi = (a,0) + (b,0) \cdot (0,1) = (a,0) + (b*0 - 0*1, b*1 + 0*0) = (a,0) + (0,b) = (a,b)$$
\end{customproof}

\begin{definition}[Conjugate]
\

If $z = a+bi$ then the \textbf{complex conjugate} of $z$, $\bar{z} = a-bi$. 

Also, $\text{Re}(z) = a$ and $\text{Im}(z) = b$
\end{definition}

\begin{theorem}
\

Let $z$ and $w$ be complex numbers. Then,

\begin{enumerate}
    \item $\bar{z+w} = \bar{z} + \bar{w}$
    \item $\bar{z \cdot w} = \bar{z} \cdot \bar{w}$
    \item $z + \bar{z} = 2\text{Re}(z)$ and $z - \bar{z} = 2i\text{Im}(z)$
    \item $z\cdot \bar{z}$ is real and positive, except when $z = 0$
\end{enumerate}
\end{theorem}

\begin{customproof}
\

Let $z = a+bi$ and $w = c+di$.

\begin{enumerate}
    \item $\bar{z} + \bar{w} = a-bi + c - di = (a+c) - (bd)i = \bar{z+w}$
    \item $\bar{z} \cdot \bar{w} = (a-bi) * (c-di) = ac-bd - (ad+bc)i = \bar{z \cdot w}$
    \item $z + \bar{z} = a+bi + a - bi = 2a = 2\text{Re}(z)$ and $z - \bar{z} = a+bi - a + bi = 2bi = 2i \text{Im}(z)$
    \item $z \cdot \bar{z} = (a+bi) * (a-bi) = a*a - b^2i^2 + abi - abi = a^2 + b^2$
\end{enumerate}
\end{customproof}

\begin{definition}[Absolute Value of a Complex Number]
\

The \textbf{absolute value} of a complex number $\vert z \vert = (z \cdot \bar{z})^{\frac{1}{2}}$ 
\end{definition}

\begin{theorem}
\

\begin{enumerate}
    \item $\vert z\vert > 0$ unless $z = 0$, then $\vert z\vert =0$
    \item $\vert \bar{z}\vert  = \vert z\vert$
    \item $\vert z \cdot w\vert = \vert z\vert \vert w\vert$
    \item $\vert \text{Re}(z)\vert \leq \vert z\vert$
    \item $\vert z+w\vert \leq \vert z\vert + \vert w\vert$             
\end{enumerate}
\end{theorem}

\begin{customproof}
\

\begin{enumerate}
    \item Since $z \cdot \bar{z}$ is always non-negative, the square root of it will also be non-negative.
    \item $ \vert \bar{z} \vert = (\bar{z} \cdot \bar{\bar{z}})^{\frac{1}{2}}  =  (\bar{z} \cdot z)^{\frac{1}{2}}  = (z \cdot \bar{z})^{\frac{1}{2}} = \vert z\vert$
    \item $\vert z \cdot w\vert = ((z\cdot w) \cdot (\bar{z\cdot w}))^\frac{1}{2} = (z \cdot \bar{z})^\frac{1}{2} * (w \cdot \bar{w})^\frac{1}{2} = \vert z\vert \vert w\vert$   
    \item $\vert z\vert = (a^2 + b^2)^\frac{1}{2}$, and this is always larger than $\vert \text{Re}(z) \vert = (a^2)^\frac{1}{2} = a$
    \item $\vert z+w \vert  = ((a+c)^2 + (b+d)^2)^\frac{1}{2}$, and this is always smaller than $\vert z \vert + \vert w \vert = (a^2 + b^2)^\frac{1}{2} + (c^2 + d^2)^\frac{1}{2}$
\end{enumerate}
\end{customproof}

\begin{theorem}
\

If we have complex numbers $a_1, ... a_n$ and $b_1, ... b_n$, then $\vert \sum_{j=1}^n a_jb_j \vert^2 \leq \vert  \sum_{j=1}^n \vert a_j\vert^2 \sum_{j=1}^n \vert b_j\vert^2$
\end{theorem}

\begin{customproof}
\

Let $A = \sum \vert a_j \vert^2$, $B = \sum \vert b_j \vert^2$ and $C = \sum a_jb)j$

Then, start with $$\sum \vert Ba_J - Cb_j \vert^2 = \sum (Ba_j - Cb_j)(B\bar{a}_j - \bar{Cb_j})$$by the property that $z \cdot \bar{z} = \vert z \vert^2$

Then, we can multiply to get $$B^2 \sum \vert a_j \vert ^2 - B\bar{C} \sum a_j \bar{b}_j - BC\sum \bar{a}_j b_j + \vert C \vert^2 \sum \vert b_j \vert^2$$

We cancel the two $BC$ terms (we can for some reason), to get $$B^2A - B \vert C \vert^2$$

$$B(AB- \vert C\vert^2)$$Since we know all the terms in the initial sum are nonnegative, our result can't be negative either. Since $B$ is postitive, this implies that $AB \geq \vert C \vert^2$ which is what we want.
\end{customproof}

\end{document}