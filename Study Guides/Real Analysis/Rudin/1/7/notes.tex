\documentclass{article}
\usepackage{alexconfig}
\title{Rudin Chapter 1: Euclidean Spaces}

\begin{document}
\maketitle

\begin{definition}[Euclidean Spaces]
\

\textbf{Euclidean spaces} $R^k$ are sets of $k$-tuples $a = (a_1, a_2, ..., a_k)$ where each $a_i$ is a real number. The tuples $a$ are called points, or vectors, and obey the following rules:

Let $a = (a_1, a_2, ..., a_k)$ and $b = (b_1, b_2, ..., b_k)$ and let $\alpha$ be a real number.

\begin{enumerate}
    \item $a+b = (a_1 + b_1, a_2 + b_2, ..., a_k+b_k)$
    \item $\alpha a = (\alpha * a_1, \alpha * a_2, ..., \alpha * a_k)$
\end{enumerate}
\end{definition}

\begin{definition}[Inner Product]
\

The \textbf{inner product} of two vectors $a \cdot b$ is: $$a \cdot b = \sum_{j=1}^k a_j* b_j$$ 
\end{definition}

\begin{definition}[Norm]
\

The \textbf{norm} of a vector is: $$\vert x \vert = (x \cdot x)^\frac{1}{2} = (\sum_{j=1}^k x_j * x_j)^{\frac{1}{2}}$$
\end{definition}

\begin{theorem}
\

\begin{enumerate}
    \item $\vert x\vert \geq 0$
    \item $\vert x\vert = 0$ iff $x = 0$
    \item $\vert \alpha x \vert = \vert \alpha \vert \vert x\vert$
    \item $\vert x \cdot y\vert \leq \vert x\vert * \vert y\vert$
    \item $\vert x+y\vert \leq \vert x\vert + \vert y\vert$
    \item $\vert x-z\vert \leq \vert x - y\vert + \vert y-z\vert$                
\end{enumerate}
\end{theorem}

\begin{customproof}
\

\begin{enumerate}
    \item Since we are squaring every component and then summing them, the squares are always nonzero and therefore, the sum of them will be nonzero as well.
    \item The only way to have a sum of zero is if each square of each component is zero, which can only happen if each component is zero.
    \item $ \vert \alpha x\vert = (\sum (\alpha x_j)^2)^\frac{1}{2} = (\alpha^2 \sum x_j^2 )^\frac{1}{2} = \vert \alpha \vert * \vert x \vert$
    \item If we look at the Schwarz Inequality which we proved in the last section, and imagine all the complex numbers are just sets of real components of vectors $a,b$, then the left side is the absolute value of the dot product, and the right side is the product of the norms, then, this inequality holds.
    \item $\vert x + y\vert^2 = (x+y) \cdot (x+y) = (x \cdot x) + 2 (x \cdot y) + (y \cdot y) \leq \vert x\vert^2 + 2 \vert x\vert \vert y\vert + \vert y\vert^2$ (by our previous inequality)
    \item Do the prior but replace $x$ with $x-y$ and $y$ with $y-z$            
\end{enumerate}
\end{customproof}

\end{document}