\documentclass{article}
\usepackage{alexconfig}
\title{Rudin Chapter 1: Exercises}

\begin{document}
\maketitle

\begin{exercise}[1]
\

If $r$ is rational and $r \neq 0$ and $x$ is irrational, show that $r+x$ and $rx$ are irrational.
\end{exercise}

\begin{solution}
\

We do a proof by contradiction. Suppose $r$ is rational, $x$ is irrational, and $r+x$ is rational. Let $r = \frac{a}{b}$ for some integers $a,b$ and $b \neq 0$. Then, we can express $$r+x = \frac{q}{p}$$for integers $q,p$ and $p \neq 0$. Then, $$\frac{a}{b} + x = \frac{q}{p}$$and we subtract over to get $$x = \frac{q}{p} - \frac{a}{b}$$Since the rationals form a field, rational numbers are closed under addition and subtraction, so therefore, $x$ is rational, which is a contradiction.

Likewise, if we assume that $rx$ is rational, we can say $$rx = \frac{q}{p}$$for some integers $q,p$ and nonzero $p$. Then, $$x = \frac{qb}{pa}$$and can be expressed as a rational number. The denominator is nonzero beacuse $r \neq 0$, so $a \neq 0$ and $p \neq 0$ either.
\end{solution}

\begin{exercise}[2]
\

Prove there is no rational number whose square is $12$.
\end{exercise}

\begin{solution}
\

Assume that there is some rational number $\frac{p}{q}$ where $(\frac{p}{q})^2 = 12$ and $p, q$ are coprime (we can say this without loss of generality because every rational number has some form where the numerator and denominator are coprime). Then, $$p^2 = 12 q^2$$this tells us that $p^2$ must be divisible by $12$. The only way this can happen is if $p$ is divisible by $2$ and $3$ (it must get all the prime factors of $12$ which are $2,2,3$ from one number squared, so the very minimum is $2,3$). 

Then, $p^2$ is divisible by $36$, so $12q^2$ is divisible by $36$, so $q^2$ must be divisible by $3$. The only way this can happen is if $q$ is divisible by $3$, but this is impossible as we said that $p$ and $q$ are coprime, so this is a contradiction. 
\end{solution}

\begin{exercise}
\

Derive the following true statements from the field axioms:

\begin{enumerate}
    \item If $x\neq 0$ and $xy= xz$ then $y=z$
    \item If $x\neq 0$ and $xy = x$ then $y=1$
    \item If $x \neq 0$ and $xy=1$ then $y = \frac{1}{x}$
    \item If $x\neq 0$ then $\frac{1}{\frac{1}{x}} = x$
\end{enumerate}
\end{exercise}

\begin{solution}
\

\begin{enumerate}
    \item We start with $x\neq 0$ and $$xy = xz$$Then, we left multiply by $\frac{1}{x}$ to get $y = z$
    \item We start with $$xy=x$$and left multiply by $\frac{1}{x}$ to get $y = 1$
    \item $$xy=1$$and we left multiply by $\frac{1}{x}$ to get $y = \frac{1}{x}$
    \item We start with$$\frac{1}{\frac{1}{x}} = x$$and left multiply by $\frac{1}{x}$ to get $$1 = 1$$so since this statement is true, and the initial statement is logically equivalent, the initial statement is true as well.
\end{enumerate}
\end{solution}

\begin{exercise}[4]
\

Let $E$ be a nonempty subset and let $\alpha$ be a lower bound on $E$ and $\beta$ be an upper bound on $E$. Prove that $\alpha \leq \beta$
\end{exercise}

\begin{solution}
\

By definition of lower bound, $\alpha \leq e$ for all $e$ in $E$, and by definition of upper bound, $e \leq \beta$ for all $e$ in $E$. By the transitive property of order, this means that $\alpha \leq \beta$.
\end{solution}

\begin{exercise}[5]
\

Let $A$ be a nonempty set of real numbers which is bounded below and $-A$ be the set of all $-x$, for all $x$ in $A$. Prove that $\inf(A) = -\sup(-A)$
\end{exercise}

\begin{solution}
\

We do a proof by contradiction. Suppose that $\inf(A) \neq -\sup(-A)$. We have two cases. 

Case 1: $\inf(A) < -\sup(-A)$

Then, choose some $x$ where $\inf{A} < x < -\sup(A)$. This means that $-x > \sup(-A)$ meaning that for all $-a$ in $-A$, $-x > -a$. However, this implies that for all $a$ in $A$, $x < a$, which means that $x$ is a lower bound of $A$, but also, $x > \inf(A)$ which is impossible, hence a contradiction.   

Case 2: $\inf(A) > -\sup(-A)$


Then, choose some $x$ where $ -\sup(A) < x < \inf{A}$. Since $x < \inf(A)$, for all $a$ in $A$, $x < a$. This implies that for all $-a$ in $-A$, $-x > -a$, making $x$ an upper bound for $-A$, but also, $-x < \sup(A)$, which is a contradiction.
\end{solution}

\begin{exercise}[6]
\

Fix $b > 1$. 

\begin{enumerate}
    \item If $m,n,q,p$ are integers and $r = \frac{m}{n} = \frac{p}{q}$ and $n, q > 0$, then show that $(b^m)^\frac{1}{n} = (b^p)^\frac{1}{q}$, and hence, $b^r = (b^m)^{\frac{1}{n}}$.
    \item Prove that $b^{r+s} = b^rb^s$ if $r,s$ are rational.
    \item Define $B(x)$ to be the set of all $b^t$ for all rational $t \leq x$. Show that $b^r = \sup B(r)$, and hence, it is logical to define that $b^x = \sup B(x)$
    \item Show that $b^{x+y} = b^xb^y$ for all real $x,y$.
\end{enumerate}
\end{exercise}

\begin{solution}
\

\begin{enumerate}
    \item We first observe that $\frac{m}{n} = \frac{p}{q}$, and when multiplying both sides by both $n$ and $p$, we get that $mq = pn$. 

Secondly, observe that if $a,b$ are integers, then $(x^a)^b = x^{a*b}$. This is true as $(x^a)^b$ is a product of $b$ number of $x^a$, and $x^a$ is a product of $a$ number of $x$, so if we break them apart, we get $a*b$ number of $x$, which is $x^{a*b}$.

We now note that $x^\frac{1}{n} = y$ implies $x = y^n$. We start with $$(b^m)^\frac{1}{n} = (b^p)^\frac{1}{q}$$and hope to transform it into some logically equivalent statement that we know the truth value of. We can say that the above statement is logically equivalent to $$b^m = ((b^p)^\frac{1}{q})^n$$

%this is the sketchy part? is it correct to say the following?

And this is equivalent to $$(b^m)^q = (b^p)^n$$and by the observation above, this means that $$b^{m*q} = b^{p*n}$$and since $mq = pn$, we know that the equality $b^{m*q} = b^{p*n}$ holds, and so the original statement, which is logically equivalent holds.

Likewise, we know that $rn = m$ so by the same logic, we can say that $$b^r = (b^m)^\frac{1}{n}$$

\item We let $r,s$ be rational, and $b>1$, and define $r = \frac{m}{n}$ and $s = \frac{p}{q}$ where $m,n,p,q$ are integers and $n,q > 0$. Then, $$b^{r+s} = b^{\frac{m}{n} + \frac{p}{q}} = b^{\frac{mq+np}{nq}}$$By the singular root theorem, which states that $x^\frac{1}{n} = y$ implies that $x = y^n$, and the previous result, we get that $$b^{r+s} = (b^{mq+np})^\frac{1}{nq}$$and so$$(b^{r+s})^{nq} = b^{mq+np}$$

Now observe that $x^{a+b} = x^ax^b$ for integers $a,b$, since if $a$ or $b$ is greater than zero, then $x^a = x*x*...*x$ for $a$ number of times, and if either $a$ or $b$ is negative, then it will be $(x^{-1})$ for $a$ number of times, cancelling out one $x$ for each $x^{-1}$ in the product. 

So now we have $$(b^{r+s})^{nq} = b^{mq} b^{np}$$and if we use our singular root identity one more time, and the corrolary to it, which states that $(xy)^a = x^ay^a$, we get that $$b^{r+s} = (b^{mq}b^{np})^\frac{1}{nq} = b^\frac{mq}{nq} b^\frac{np}{nq} = b^\frac{m}{n} b^\frac{p}{q}$$

\item We first want to show that for $x>1$, then $a<b$ implies $x^a < x^b$. 

We start with the statement that $$x^a<x^b$$By multiplying on both sides by $x^-b$, we get $$x^{a-b} < 1$$Since we know that $a<b$, we know that $a-b < 0$, so $b-a>0$, so we have $$\frac{1}{x^{b-a}} < 1$$Since we know that $x^{b-a}$ is greater than one (since $b-a > 0$ and $x>1$), the denominator is larger than the numerator, so this fraction is indeed less than one, which is true, making the logically equivalent original statement true.

From here on out, it is simple to see that the supremum of the set $B(r)$ will be the largets $b^t$ possible, which is obtained by picking the largest $t$ possible, which happens to be $r$, so the supremum of the set $B(r)$ is $b^r$

\item Let $\alpha$ be in $B(x)$ and $\beta$ be in $B(y)$. Then, consider the set composed of elements of the form $\alpha * \beta$. The supremum of this set is $b^{x+y}$, and the supremum of the individual $B(x)$, $B(y)$ are $b^x$ and $b^y$ respectively, so we can conclude that $b^{x+y} = b^xb^y$.
\end{enumerate}
\end{solution}

\end{document}