\documentclass{article}
\usepackage{tikz}
\usepackage{parskip}
\usepackage{xcolor}
\usepackage{textcomp, gensymb}
\usepackage{pgfplots}
\usepackage{tkz-euclide}
\usepackage[bottom=0.5in,top=0.5in,right=0.5in,left=0.5in]{geometry}
\usepackage{amsmath}
\usepackage{amsfonts}
\usepackage{amssymb}
\usepackage{enumitem}
\usepackage{amsthm}
\pgfplotsset{compat=1.18}
\title{2.2: Basis of a Topology}
\author{Alex L.}
\date{\today}
\pagecolor[rgb]{0,0,0} %black
\color[rgb]{1,1,1} %white

\begin{document}
\maketitle
\textbf{Def:} If $X$ is a set, a \textbf{basis} for a topology on $X$ is a set $\mathcal{B}$ subsets of $X$, called \textbf{basis elements}, such that \begin{enumerate}
    \item For each $x \in X$, there is at least one basis element $B$ such that $x \in B$
    \item If $x$ belongs to the intersection of two basis elements, there is a third basis element that is a subset of the intersection of the first two basis elements that contains $x$. 
\end{enumerate}

\textbf{Ex:} Let $\mathcal{B}$ be the interiors of all rectangles with sides parallel to the axes of a 2D plane. Then, $\mathcal{B}$ is a basis, because every point on the plane can be enveloped by a rectangle, and the intersection of two rectangles can always contain another rectangle.

\textbf{Def:} We define a \textbf{topology $\mathcal{T}$ generated by a basis $\mathcal{B}$} in the following way: Pick a subset $U \subset X$. That subset is in $\mathcal{T}$ generated by $\mathcal{B}$, if for every $u \in U$, there was a basis element $B \in \mathcal{B}$ such that the basis element contained $u$ and the subset $U$ contained $B$, that is $x \in B$ and $B \subset U$.

Let's verify that $\mathcal{T}$ generated by $\mathcal{B}$ is actually a topology. It contains the empty set, and the entire set $X$ fulfills the conditions because basis elements by definition contain every $x \in X$, and are subsets of $X$ themselves. 

Arbitrary unions, $U = \bigcup U_a$, fulfill the criteria as well, because for every $x \in U$, there must be a $U_a$ such that $x\in U_a$ by definition of a union. Then, there must be a basis element in $U_a$ because $U_a$ is in the topology generated by $\mathcal{B}$. Then, that basis element must be in $U$ as well, by definition of a union. Therefore, $U \in \mathcal{T}$.

Finite intersections work as well, because for any $x$ in the intersection of two members of the topology generated by $\mathcal{B}$, they must be a member of at least two basis elements. By definition of basis elements, there must be another basis element existing in the intersection that also contains $x$, therefore, the intersection of elements is in the topology generated by $\mathcal{B}$

\textbf{Lemma:} Let $X$ be a set, let $\mathcal{B}$ be a basis for a topology $\mathcal{T}$ on $X$. Then $\mathcal{T}$ equals the set of all unions of elements of $\mathcal{B}$.

\textbf{Proof:} Given a collection of elements of $\mathcal{B}$, they are also elements of $\mathcal{T}$. Their union is also in $\mathcal{T}$

\textbf{Lemma:} Let $X$ be a topological space. For each open set $U \in X$ and each $x \in U$, there is an open set $C \in \mathcal{C}$ such that $x \in C \subset U$. Then, $\mathcal{C}$ is a basis on $X$.

\textbf{Proof:} For the first condition, if $X = U$, then there will be at least one basis element for every $x \in X$. For the seocnd condition, since all elements of $\mathcal{C}$ are open, there will exist another $C \in \mathcal{C}$ for all intersections of elements of $C$. 

\textbf{Lemma:} Let $\mathcal{B}$ and $\mathcal{B'}$ be bases for $\mathcal{T}$ and $\mathcal{T'}$ respectively, on $X$. Then, the following are equivalent:

\begin{enumerate}
    \item $\mathcal{T'}$ is finer than $\mathcal{T}$
    \item For each $x \in X$ and each basis element $B \in \mathcal{B}$ containing $x$, there is a basis element $B' \in \mathcal{B'}$ such that $x\in \mathcal{B}' \subset B$
\end{enumerate}

\textbf{Proof:} $2 \Rightarrow 1$ To show that $\mathcal{T'}$ is finer than $\mathcal{T}$, we need to show that every subset $U \in \mathcal{T}$ is in $\mathcal{T'}$. Since $\mathcal{B}$ generates $\mathcal{T}$, then there is a basis element $B$ such that $x \in B \subset U$. We assume statement $2$ is true, so there must be $B' \in \mathcal{B}$ such that $x \in B' \subset B \subset U$, and then, by definition, $U \in \mathcal{T'}$

$1\Rightarrow 2$: Let $B \in \mathcal{B}$, then $B \in \mathcal{T}$ and $\mathcal{T} \subset \mathcal{T'}$ so $B \in \mathcal{T'}$. Since $\mathcal{T'}$ is generated by $\mathcal{B'}$, there is a $x \in B' \subset B$. 

\textbf{Def:} If $\mathcal{B}$ is the collection of all open intervals on the real line, then the topology generated by $\mathcal{B}$ is called the \textbf{standard topology} on the real line, and we will assume real lines have this topology unless specifically stated otherwise. 

\textbf{Def:} If $\mathcal{B}$ is the set of all half open intervals $[a,b)$ on $\mathbb{R}$, then the topology generated by $\mathcal{B}$ is called the \textbf{lower limit topology} on the reals. When $\mathbb{R}$ is given the lower limit topology, denote it $\mathbb{R}_l$. 

\textbf{Def:} Let $K$ denote the set of all numbers of the form $\frac{1}{n}$ for $n \in \mathbb{Z}^+$, and let $\mathcal{B}$ be the collection of all open intervals along with all the sets of the form $(a,b)-K$. The topology generated by $\mathcal{B}$ is called the \textbf{K-topology} on the reals, and we denote it $\mathbb{R}_K$

\textbf{Lemma:} $\mathbb{R}_l$ and $\mathbb{R}_K$ are finer than the standard topology on $\mathbb{R}$, but the former two are not comparable to each other.

\textbf{Proof:} Given a basis element $(a,b)$ on the standard topology, and a point $x$ within it, we can construct a basis element for $\mathbb{R}_l$, $[x,b)$, which is strictly within $(a,b)$ and still contains $x$, therefore, $\mathbb{R}_l$ is finer than the standard topology.

Likewise, $\mathbb{R}_K$ has all the basis elements of the standard topology, and also basis elements of the form $(a,b)-K$, so the standard topology is coarser than $\mathbb{R}_K$.

\textbf{Def:} A \textbf{subbasis} $\mathcal{S}$ for a topology on $X$ is a set of subsets of $X$ whose union equals $X$. The \textbf{topology generated by the subbasis} $\mathcal{S}$ is defined to be the set of all unions of finite intersections of elements of $S$.     
\end{document}